\section{$R$-Symmetry}
$R$-symmetry is an additional continuous symmetry which extends  supersymmetry in a non-trivial way. This is actually covered in the Haag-Lopuszanski-Sohnius theorem \cite{Haag:1974qh} but often concealed.\\
This section introduces $R$-symmetry and discusses briefly a minimal viable extension of the MSSM including $R$-symmetry, the MRSSM. Thereby some general features of the model are mentioned before the focus is laid on the strongly interacting sector which is the central topic of the thesis. 


\subsection{$R$-Symmetry Transformations}
For a $N=1$ supersymmetric theory, $R$-symmetry is a global $U(1)$ symmetry. It must not be confused with $R$-parity which is a discrete $\mathbb{Z}_2$ symmetry. A continuous global symmetry implies according to Noether's theorem a conserved charge. In the case of $R$-symmetry it is called $R$-charge and one therefore refers to $R$-symmetry as $U_R(1)$.\\
The defining property of $U_R(1)$ is that the anticommuting coordinates $\theta^\alpha$ and $\overline{\theta}^{\dot{\alpha}}$ transform like
\begin{align}
&\theta \to \mathrm{e}^{i\tau}\theta, && \overline{\theta} \to \mathrm{e}^{-i\tau}\overline{\theta},
\end{align}
where $\tau$ parametrizes the transformation. This in turn implies that the supersymmetry generators (see eq. \eqref{eq:SUSYGen}) transform like
\begin{align}
& Q \to \mathrm{e}^{-i\tau} Q && \overline{Q} \to \mathrm{e}^{i\tau}\overline{Q}
\end{align}
and $R$-symmetry does not commute with supersymmetry 
\begin{align}
\left[ R,Q \right] = -Q && \left[ R,\overline{Q} \right] = \overline{Q},
\end{align}
meaning that superpartners do not have the same $R$-charge. The transformation of chiral and vector superfields reads
\begin{align}
& \hat{\Phi}(x,\theta,\overline{\theta}) \to \mathrm{e}^{ir_{\hat{\Phi}}\tau}\ \hat{\Phi}(x,\mathrm{e}^{i\tau}\theta,\mathrm{e}^{-i\tau}\overline{\theta}),\nonumber\\
&\hat{V}(x,\theta,\overline{\theta}) \to  \hat{V}(x,\mathrm{e}^{i\tau}\theta,\mathrm{e}^{-i\tau}\overline{\theta}),
\end{align}
i.e. the chiral superfields transform with arbitrary $R$-charge where the $R$-charge of vector superfield is restricted to be zero by the condition that they are real. If one inserts the component decomposition eq. \eqref{eq:superfielddecomp} of the superfields one can read off the $R$-charges of the component fields, see table \ref{tab:Rcharges}.
\begin{table}
\begin{center}
\begin{tabular}{c|c?c|c?c|c}
\multicolumn{2}{c?}{Superfield} & \multicolumn{2}{c?}{Boson} & \multicolumn{2}{c}{Fermion} \\
\hlinewd{2pt}
$\hat{\Phi}$ & $r_{\hat{\Phi}}$ & $A$ & $r_{\hat{\Phi}}$ & $\psi$ & $r_{\hat{\Phi}}-1$\\
$\hat{V}$ & 0 & $v^\mu$ & 0 & $\lambda$ & $+1$
\end{tabular}
\caption{This table shows the R-charges of a generic chiral and vector superfield.}\label{tab:Rcharges}
\end{center}
\end{table}

%\begin{table}
%\begin{center}
%\begin{tabular}{c|c||c|c||c|c}
%\multicolumn{2}{c||}{superfield} & \multicolumn{2}{c||}{boson} & \multicolumn{2}{c}{fermion} \\
%\hhline{=|=#=|=#=|=}
%$\hat{\Phi}$ & $r_{\hat{\Phi}}$ & $A$ & $r_{\hat{\Phi}}$ & $\psi$ & $r_{\hat{\Phi}}-1$\\
%$\hat{V}$ & 0 & $v^\mu$ & 0 & $\lambda$ & $+1$
%\end{tabular}
%\caption{This table shows the$R$-charges of a generic chiral and vector superfield.}
%\end{center}
%\end{table}



\subsection{The Minimal $R$-Symmetric Supersymmetric Standard Model}
If one imposes $R$-symmetry upon the MSSM one is faced with a certain arbitrariness, i.e. the choice of the $R$-charges of the chiral superfields. A minimal $R$-symmetric extension of the MSSM is the minimal $R$-symmetric supersymmetric Standard Model (MRSSM)\cite{Kribs:2007ac}. In this model the $R$-charges are chosen in such a way, that every Standard Model particle has $R$-charge zero. Following this one obtains the $R$-charges of all particles which are summed up in table \ref{tab:R_charges_MRSSM}\cite{Diessner:2014ksa}.
\begin{table}[H]
\begin{center}
\begin{tabular}{c?c|c?c|c?c|c}
Field & \multicolumn{2}{c?}{Superfield} & \multicolumn{2}{c?}{Boson} & \multicolumn{2}{c}{Fermion} \\
\hlinewd{2pt}
gauge Vector & $\hat{V}_s^a$, $\hat{V}_w^a$, $\hat{V}_Y$ & 0 & $G^a_\mu$, $W^a_\mu$, $B_\mu$ & 0 & $\lambda_s^a$, $\lambda_w^a$, $\lambda_Y$ & $+1$\\
matter & $\hat{L}$, $\hat{E}$ & $+1$ & $\tilde{l}_l$, $\tilde{e}^\dagger_R$ & $+1$ & $l_L$, $e_R$ & 0\\
 & $\hat{Q}$, $\hat{D}$, $\hat{U}$ & $+1$ & $\tilde{q}_L$, $\tilde{d}^\dagger_R$, $\tilde{u}^\dagger_R$ & $+1$ & $q_L$, $d_R$, $u_R$ & 0\\
$H$-Higgs & $\hat{H}_{d,u}$ & 0 & $H_{d,u}$ & 0 & $\tilde{H}_{d,u}$ & $-1$\\
\hline
$R$-Higgs & $\hat{R}_{d,u}$ & $+2$ & $R_{d,u}$ & $+2$ & $\tilde{R}_{d,u}$ & $+1$\\
adjoint chiral & $\hat{O}$, $\hat{T}$, $\hat{S}$ & 0 & $\sigma^a$, $\omega^a$, $\rho$ & 0 & $\chi_s^a$, $\chi_w^a$, $\chi_Y$ & $-1$
\end{tabular}
\caption{This table lists the $R$-charges of all superfields and their components in the MRSSM. The fields of the $R$-Higgs and the adjoint chiral superfields are not present in the MSSM.}\label{tab:R_charges_MRSSM}
\end{center}
\end{table}

%\begin{table}[H]
%\begin{center}
%\begin{tabular}{c||c|c||c|c||c|c}
%Field & \multicolumn{2}{c||}{Superfield} & \multicolumn{2}{c||}{Boson} & \multicolumn{2}{c}{Fermion} \\
%\hhline{=#=|=#=|=#=|=}
%Gauge Vector & $\hat{V}_s^a$, $\hat{V}_w^a$, $\hat{V}_Y$ & 0 & $G^a_\mu$, $W^a_\mu$, $B_\mu$ & 0 & $\lambda_s^a$, $\lambda_w^a$, $\lambda_Y$ &+1\\
%Matter & $\hat{L}$, $\hat{E}$ & 0 & $\tilde{l}_l$, $\tilde{e}^\dagger_R$ & +1 & $l_L$, $e_R$ & 0\\
% & $\hat{Q}$, $\hat{D}$, $\hat{U}$ & +1 & $\tilde{q}_L$, $\tilde{d}^\dagger_R$, $\tilde{u}^\dagger_R$ & +1 & $q_L$, $d_R$, $u_R$ & 0\\
%$H$-Higgs & $\hat{H}_{d,u}$ & 0 & $H_{d,u}$ & 0 & $\tilde{H}_{d,u}$ & -1\\
%\hline
%$R$-Higgs & $\hat{R}_{d,u}$ & +2 & $R_{d,u}$ & +2 & $\tilde{R}_{d,u}$ & +1\\
%Adjoint Chiral & $\hat{O}$, $\hat{T}$, $\hat{S}$ & 0 & $\sigma^a$, $\omega^a$, $\rho$ & 0 & $\tilde{\chi}_s^a$, $\tilde{\chi}_w^a$, $\tilde{\chi}_Y$ & -1
%\end{tabular}
%\caption{This table lists the$R$-charges of all superfields and their components in the MRSSM. The fields of the $R$-Higgs and the adjoint chiral superfields are not present in the MSSM}\label{tab:R_charges_MRSSM}
%\end{center}
%\end{table}
The gauge, matter and $H$-Higgs fields are the fields of the MSSM. Below the thin horizontal line, one finds the fields which are not present in the MSSM, i.e. the $R$-Higgs and adjoint chiral fields. These occur for the following reason.\\
In the MSSM the gauginos are Majorana particles. Their mass terms  reads in two-spinor notation
\begin{align}
\mathcal{L}_{\mathrm{Majorana\ mass}} = -m\lambda\lambda + h.c.
\end{align}
which is not $R$-invariant because the Weyl fermion $\lambda$ has $R$-charge $+1$. Giving no mass to the gauginos is phenomenologically not possible. The only way to account for a fermion mass is to write down a Dirac mass term
\begin{align}
\mathcal{L}_{\mathrm{Dirac\ mass}} = -m \chi\lambda + h.c.,
\end{align}
where $\chi$ is another Weyl spinor. This means that the four-spinor of every gaugino is a Dirac  instead of a Majorana spinor, e.g. for the gluino:
\begin{align}
& \tilde{g}_\mathrm{MSSM} = \begin{pmatrix}
-i \lambda_s \\
+i \overline{\lambda}_s
\end{pmatrix} && \tilde{g}_\mathrm{MRSSM} = \begin{pmatrix}
-i \lambda_s \\
+i \overline{\chi}_s
\end{pmatrix}.
\end{align}
In order to get a $R$-symmetric mass term on has to choose the $R$-charge of the new Weyl spinor $\chi$ to be the opposite of $\lambda$.\\
This explains the necessity of enlarging the field content if one imposes $R$-symmetry.\\
Of course the new Weyl-spinor $\chi$ must have also a superpartner. One chooses this superpartner to be a scalar, i.e. the additional Weyl fermion comes from a chiral superfield. In order to maintain gauge invariance, this chiral superfield has to transform in the adjoint representation, hence the name adjoint chiral in table \ref{tab:R_charges_MRSSM}. To fix notation the component decomposition of the eight chiral supermultiplets assosiated to the gluons is given by
\begin{align}
&\hat{O}^a(x, \theta, \overline{\theta}) = \sigma^a + \sqrt{2}\theta i\chi^a_s + \hdots && a = 1,\hdots,8.
\end{align}
The scalar components $\sigma^a$ are referred to as scalar gluons and the Weyl spinors $\chi^a$ are called octinos.\\
The same argument as for the adjoint chiral superfields explains the existence of additional Higgs-superfields which are referred to as $R$-Higgs fields.\\
Apart from these extensions $R$-symmetry also forbids terms which are allowed by supersymmetry. For the above choice of $R$-charges the $\mu$-term in \eqref{eq:W_MSSM} and the $A$-terms in the last line of \eqref{eq:L_soft} are absent in the MRSSM. Inspecting eq. \eqref{eq:off_diagonal} implies that there is no squark mixing in the MRSSM. As a consequence terms which allow flavor violating processes like $\mu \to e \gamma$ in the MSSM are suppressed in the MRSSM\cite{Kribs:2007ac}. Quite interestingly Dirac gauginos can be heavier than Majorana gauginos without being less natural in view of the hierarchy problem\cite{Nelson:2002ca, Kribs:2012gx, Hardy:2013ywa}. Finally the production cross section of strongly interacting particles is suppressed\cite{Kribs:2012gx, Kribs:2013oda}. It is this last feature which is going to be observed further in this thesis. To this end the colored sector of the MRSSM is now considered in more detail.


\subsection{The $R$-Symmetric Supersymmetric Quantum Chromodynamics}
The subject of this thesis is the phenomenology of the strongly coupling sector of the MRSSM. The $R$-symmetric supersymmetric quantum chromodynamics (RSQCD) is therefore considered closer. Its Lagrangian for an arbitrary massless quark reads
\begin{align}
\mathcal{L}_{\mathrm{RSQCD}} = &\int\mathrm{d}^4\theta\ \left( \hat{\overline{Q}}_L \mathrm{e}^{2g_s\hat{V}_s} \hat{Q}_L + \hat{\overline{Q}}_R \mathrm{e}^{-2g_s\hat{V}^T_s} \hat{Q}_R + \hat{\overline{O}} \mathrm{e}^{2g_s\hat{V}^{\mathrm{fund}}_s} \hat{O}\right)\nonumber\\
+& \left( \int \mathrm{d}^2\theta \frac{1}{16g_s^2} \hat{W}_s^{a\alpha}\hat{W}^a_{s\alpha} + h.c. \right) + \mathcal{L}_{\mathrm{soft}}\label{eq:L_super_RSQCD}
\end{align}
where $\hat{Q}_L$ and $\hat{Q}_R$ refer to the left- and right-handed superfield of the quark. Note that each field in the first line of eq. \eqref{eq:L_super_RSQCD} transforms in a different representation of $SU_C(3)$, i.e. the fundamental, the antifundamental and the adjoint one. From now on $T^a$ stands for the fundamental representation of the generators of $SU(3)_C$. In terms of components fields these expressions are given by
\begin{align}
\int\mathrm{d}^4\theta\ \hat{\overline{Q}}_L \mathrm{e}^{2g_s\hat{V}_s} \hat{Q}_L =\ & F_L^\dagger F_L + (D_\mu \tilde{q}_L)^\dagger (D^\mu \tilde{q}_L) + \overline{q}_L \overline{\sigma}^\mu i D_\mu q_L\nonumber\\
&-\sqrt{2}g_s \left( -i (\tilde{q}_L^\dagger T^a q_L ) \lambda^a + i \overline{\lambda}^a (\overline{q}_L T^a \tilde{q}_L) \right) + g_s\tilde{q}_L^\dagger T^a D^a \tilde{q}_L,\label{eq:RSQCD_Feynmanrules1}\\
\int\mathrm{d}^4\theta\ \hat{\overline{Q}}_R \mathrm{e}^{-2g_s\hat{V}^T_s} \hat{Q}_R =\  & F_R^\dagger F_R + (D_\mu \tilde{q}_R)^\dagger (D^\mu \tilde{q}_R) + \overline{q}_R \overline{\sigma}^\mu i D_\mu q_R\nonumber\\
&+\sqrt{2}g_s \left( -i (\tilde{q}_R T^{\ast a} q_R ) \lambda^a + i \overline{\lambda}^a (\overline{q}_R T^{\ast a} \tilde{q}_R^\dagger) \right) - g_s \tilde{q}_R T^{\ast a} D^a \tilde{q}_R^\dagger,\label{eq:RSQCD_Feynmanrules2}\\
\int\mathrm{d}^4\theta\ \hat{\overline{O}} \mathrm{e}^{2g_s\hat{V}^{\mathrm{fund}}_s} \hat{O} =\  & F_O^\dagger F_O + (D_\mu \sigma^a)^\dagger (D^\mu \sigma^a) + \overline{\chi} \overline{\sigma}^\mu i D_\mu \chi\nonumber\\
&-\sqrt{2}g_s \left( -i (\sigma_b^\dagger (-if_{abc}) (-i\chi^c) ) \lambda^a + i \overline{\lambda}^a (i\overline{\chi}_b (+if_{abc}) \sigma^{c\dagger}) \right)\nonumber\\
&-ig_s\sigma^{b^\dagger} f^{abc}D^a\sigma^c,
\end{align}
where in the gauge covariant derivative $D_\mu = \partial_\mu +ig_sT^aG^a_\mu$ the generator $T^a$ needs to be replaced by $-T^{\ast a}$ or $-if^{abc}$ if applied to a field transforming in the antifundamental or adjoint representation respectively.\\
The soft breaking Lagrangian accounts for the squark, gaugino and scalar gluon masses. These mass terms arise from a hidden sector spurion. For gauginos the D-type spurion is given by $\hat{W}_\alpha^\prime = \theta_\alpha D$ and mediates super symmetry breaking at the mediation scale $M$: $\int\mathrm{d}\theta^2\frac{\hat{W}_\alpha^\prime}{M}W_s^\alpha \hat{O}$. After integrating out the spurion one obtains\cite{Fox:2002bu, Diessner:2015bna}
\begin{align}
\mathcal{L}_{\mathrm{soft}} =\ & -\frac{m_{\tilde{q}}^2}{2}(|\tilde{q}_L|^2 + |\tilde{q}_R|^2),\nonumber\\
& -m_{\sigma}^2\left|\sigma^{a}\right|^2 - m_g(\lambda\chi -\sqrt{2}D^a \sigma^a + h.c.)
\end{align}
where the complex scalar gluons which are also referred to as sgluons
\begin{align}
\sigma = \frac{\phi_0 + i\sigma_0}{\sqrt{2}},
\end{align}
constitute of two real ones with different masses.
The equations of motion for the auxiliary fields are
\begin{align}
D^a =& -g_s \tilde{q}_L^\dagger T^a \tilde{q}_L + g_s \tilde{q}_R T^{\ast a} \tilde{q}_R^\dagger + ig_s\sigma^{\dagger b}f^{abc}\sigma^c -\sqrt{2}m_g(\sigma^a + \sigma^{\dagger a}),\\
F_i =&\ 0 \hspace{3cm} \mathrm{for} \hspace{3cm} i = L,R,O
\end{align}
where $D^a$ is still real as the purely imaginary parts do not contribute by virtue of the antisymmetry of the structure constants. After eliminating the auxiliary fields the complete Lagrangian reads in four.spinor notation\footnote{How a four-spinor is composed of Weyl-spinors is given in the Appendix \ref{sec:2spinor_notation}.}
\begin{align}
\mathcal{L}_{\mathrm{RSQCD}} = & |D_\mu \sigma|^2 + |D_\mu \tilde{q}_R|^2 + |D_\mu \tilde{q}_L|^2 + \overline{q}i\slashed{D}q + \overline{\tilde{g}}^ai\slashed{D}P_L\tilde{g}^a + \overline{\tilde{g}}^ai\slashed{D}P_R\tilde{g}^a - \frac{1}{4} (F_a^{\mu\nu})^2\nonumber\\
& -\sqrt{2}g_s \left( \overline{\tilde{g}}^a P_R (q^C T^a \tilde{q}_L) + (\tilde{q}_L^\dagger T^a \overline{q}^C)P_L\tilde{g}^a \right)\nonumber\\
& +\sqrt{2}g_s \left( \overline{\tilde{g}}^a P_R (q T^{\ast a} \tilde{q}_R^\dagger) + (\tilde{q}_R T^{\ast a} \overline{q})P_L\tilde{g}^a \right) \nonumber\\
& -\sqrt{2}g_s \left( \overline{\tilde{g}}^a P_R (\tilde{g}^b (if^{abc}) \sigma^c) + (\sigma^{\dagger b} (-if^{abc}) \overline{\tilde{g}}^c)P_L\tilde{g}^a \right) \nonumber\\
& -\frac{m_{\tilde{q}}^2}{2}(|\tilde{q}_L|^2 + |\tilde{q}_R|^2) -m_{\sigma}^2\left|\sigma^a\right|^2  - m_{\tilde{g}} \overline{\tilde{g}}^a\tilde{g}^a\nonumber\\
& -\frac{1}{2}\left( g_s \tilde{q}_L^\dagger T^a \tilde{q}_L - g_s \tilde{q}_R T^{\ast a} \tilde{q}_R^\dagger - ig_s\sigma^{\dagger b}f^{abc}\sigma^c + \sqrt{2}m_{\tilde{g}}(\sigma^a + \sigma^{\dagger a}) \right)^2\label{eq:L_RSQCD}
\end{align}
Observe that there is no three sgluon vertex, because of the antisymmetry of the structure constants $f^{abc}$. As already alluded to there are two distinct sgluons with different masses. The mass of the pseudoscalar $\sigma_0^a$ is given by the soft breaking parameter $m_{\sigma_0} = m_\sigma$ whereas the mass of the scalar $\phi_0$ is given by 
\begin{align}
m_{\phi_0} = \sqrt{m_{\sigma_0}^2 + 4 m_{\tilde{g}}^2}.
\end{align}
The Feynman rules inferred from this Lagrangian are given in Appendix \ref{sec:FeynmanRules}.
\newpage
