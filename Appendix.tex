\section{Appendix}


\subsection{System of Units and Metric}
In this thesis the natural units are used, i.e. $c= \hbar (= k_B) = 1$. Furthermore the Minkowski metric is chosen to be
\begin{align}
g^{\mu\nu} = \mathrm{diag}(1, -1, -1, -1).
\end{align}


\subsection{Constants of the Colour Algebra $SU(N)$}
[Marina von Steinkirch]\\
Tha Casimir operator $C(R)\mathbbm{1}$ of a semi-simple Lie algebra in the irreducible representation $R$ is given by
\begin{align}
g^{ab} T^a(R)T^b(R) = C(R)\mathbbm{1} \label{eq:Casimir}
\end{align}
where $T^a(R)$ is the $a$-th generator of the matrix valued representation $R$, $g^{ab}$ is the metric of group, $C(R)$ is the Quadratic Casimir invariant of the representation $R$ and $\mathbbm{1}$ is the identity in the representation space.\\
Apart from $C(R)$ it is common to define the Dynkin-Index $T(R)$:
\begin{align}
\mathrm{Tr}\left[T^a(R)T^b(R)\right] = T(R)\delta^{ab}.
\end{align}
The two constants are connected by
\begin{align}
C(R) \cdot \mathrm{dim}(R) = T(R) \cdot \mathrm{dim}(G)
\end{align}
where $\mathrm{dim}(G)$ is the dimension of the group and $\mathrm{dim}(R)$ is the dimension of the irreducible representation $R$.\\
In the case of $SU(N)$ one has a diagonal metric $g^{ab} = \delta^{ab}$ and therefore \ref{eq:Casimir} turns to
\begin{align}
\sum_a (T^a(R))^2 = C(R)\mathbbm{1}_{\mathrm{dim}(R) \times \mathrm{dim}(R)}
\end{align}
and one can write down the following useful formulae for the fundamental representation $R=F$: $T^a_{ij} = \frac{\lambda^a_{ij}}{2}$ and the adjoint representation $R=A$: $(T^a_{ij})^{adj} = -if_{aij}$
\begin{align}
T^a_{ik} T^a_{kj} &= C(F)\mathbbm{1}_{ij} && \mbox{with}\ C(F) = \frac{N^2-1}{2N} = \frac{4}{3}\nonumber\\
f^{abc} f^{dbc} &= C(A)\delta^{ad} && \mbox{with}\ C(A) = N = 3\nonumber\\
\mathrm{Tr}\left[T^aT^b\right] &= T(F)\delta^{ab} && \mbox{with}\ T(F) = \frac{1}{2}\label{eq:DynkinIndex}
\end{align}
where $\lambda^a_{ij}$ are for $N_c = 3$ the Gell-Mann matrices and $f_{abc}$ are the structure constants of $SU(N_c)$.
%\begin{tabular}{c|c|c}
%Quadratic Casimir Invariant and $T(R)$ & $SU(N)_c$ & $SU(3)_c$\\
%\hline
%$C_2(3)$ & $\frac{N^2-1}{2N}$ & $\frac{4}{3}$\\
%$C_2(8)$ & $N$ & $3$\\
%$T(3)$ & $\frac{1}{2}$ & $\frac{1}{2}$ \\
%$T(8)$ & $N$ & $3$ 
%\end{tabular}



\subsection{Weyl basis and 2-spinor notation}\label{sec:2spinor_notation}
As representation of the $\gamma$-matrices the Weyl or chiral representation is chosen:
\begin{align}
&\gamma^\mu = \begin{pmatrix}
0 && \sigma^\mu\\
\overline{\sigma}^\mu && 0
\end{pmatrix} &&
\gamma_5 = \begin{pmatrix}
-\mathbbm{1}_2 && 0\\
0 && \mathbbm{1}_2
\end{pmatrix}
\end{align}
with
\begin{align}
&\sigma^\mu = \begin{pmatrix}
\mathbbm{1}_2 && \sigma^i
\end{pmatrix}, &&
\overline{\sigma}^\mu = \begin{pmatrix}
\mathbbm{1}_2 && -\sigma^i
\end{pmatrix},\label{eq:Pauli}
\end{align}
where $\sigma^i$ are the Pauli matrices and $\mathbbm{1}_n$ is the $n \times n$ unit matrix. The left and right handed projectors are then given by
\begin{align}
P_L = \frac{1}{2}(\mathbbm{1}_4 - \gamma_5)\begin{pmatrix}
\mathbbm{1}_2 && 0\\
0 && 0
\end{pmatrix} && 
P_R = \frac{1}{2}(\mathbbm{1}_4 + \gamma_5)\begin{pmatrix}
0 && 0\\
0 && \mathbbm{1}_2
\end{pmatrix}\label{eq:projectors}
\end{align}
The representation of generators of the Lorentz group on 4-spinor space are composed of the above matrices. Because of the block form of those it is not surprising that the representation on 4-spinor space is reducible to two representations on 2-spinor (Weyl spinor) spaces. It is therefore sensible to decompose a 4 spinor into a left and a right handed Weyl spinor\footnote{The projectors in the chiral basis (eq. \ref{eq:projectors}) explain the names left and right handed Weyl spinors.}
\begin{align}
\Psi = \begin{pmatrix}
\psi_\alpha \\
\overline{\chi}^{\dot{\alpha}}
\end{pmatrix}
\end{align}
where $\alpha,\dot{\alpha} \in \left\{ 1,2 \right\}$. Left handed Weyl spinors are labeled with undotted and right handed Weyl spinors with dotted indices. One distinguishes 4 different Weyl spinors:
\begin{align}
\psi^\alpha, \hspace{1cm} 
\overline{\psi}^{\dot{\alpha}} = (\psi^\alpha)^\ast, \hspace{1cm} \psi_\alpha = \epsilon_{\alpha\beta}\psi^\beta, \hspace{1cm} \mathrm{and} \hspace{1cm} \psi_{\dot{\alpha}} = \epsilon_{\dot{\alpha}\dot{\beta}}\psi^{\dot{\beta}} = (\psi_\alpha)^\ast,
\end{align}
where $^\ast$ denotes complex conjugation and indices are lowered with the antisymmetric $\epsilon_{\alpha\beta}$ ($\epsilon_{\dot{\alpha}\dot{\beta}}$), which obeys
\begin{align}
\epsilon^{\alpha\beta} = \epsilon_{\beta\alpha},\ \epsilon^{\dot{\alpha}\dot{\beta}} = \epsilon_{\dot{\beta}\dot{\alpha}} \hspace{3cm} \mathrm{and} \hspace{3cm} \epsilon_{12} = \epsilon_{\dot{1}\dot{2}}= 1.
\end{align}
By virtue of the antisymmetry of $\epsilon$ one finds the Lorentz invariant products:
\begin{align}
\psi\chi &:= \psi^\alpha\chi_\alpha = -\chi_\alpha\psi^\alpha = \chi^\alpha\psi_\alpha = \chi\psi,\nonumber\\
\overline{\psi}\overline{\chi} &:= \overline{\psi}_{\dot{\alpha}} \overline{\chi}^{\dot{\alpha}} = -\overline{\chi}^{\dot{\alpha}}\overline{\psi}_{\dot{\alpha}} = \overline{\chi}_{\dot{\alpha}}\overline{\psi}^{\dot{\alpha}} = \overline{\chi}\overline{\psi} \label{eq:Weylspinorproduct}
\end{align}
To make the index structure of the Pauli matrices explicit one writes $\sigma^\mu_{\alpha\dot{\alpha}}$ and $\overline{\sigma}^{\mu\dot{\alpha}\alpha}$ for the formulae in \ref{eq:Pauli}. For the definition of the supersymmetry algebra in section \ref{sec:SUSYalgebra} the representation of the generators of the Lorentz group on the left and right handed Weyl spinor space are introduced:
\begin{align}
\frac{1}{2}(\sigma^{\mu\nu})_\alpha^{\ \beta} := \frac{i}{4}(\sigma^\mu\overline{\sigma}^\nu - \sigma^\nu\overline{\sigma}^\mu)_\alpha^{\ \beta}\nonumber\\
\frac{1}{2}(\overline{\sigma}^{\mu\nu})^{\dot{\alpha}}_{\ \dot{\beta}} := \frac{i}{4}(\overline{\sigma}^\mu\sigma^\nu - \overline{\sigma}^\nu\sigma^\mu)^{\dot{\alpha}}_{\ \dot{\beta}}
\end{align}

 
With the definition of bared and charge conjugated 4-spinors\footnote{$\Psi^T$ denotes the transpose of the spinor $\Psi$ and $\Psi^\dagger$ is the Hermitian adjoint of $\Psi$.}
\begin{align}
&\overline{\Psi} := \Psi^\dagger \gamma^0, && \Psi^C := i\gamma^2\gamma^0 \overline{\Psi}^T  
\end{align}
one obtains:
\begin{align}
\Psi &= \begin{pmatrix}
\psi_\alpha \\
\overline{\chi}^{\dot{\alpha}}
\end{pmatrix}, && \ \ 
\overline{\Psi} = \begin{pmatrix}
\chi^\alpha & \overline{\psi}_{\dot{\alpha}}
\end{pmatrix},\nonumber\\
\Psi^C &= \begin{pmatrix}
\chi_\alpha \\
\overline{\psi}^{\dot{\alpha}}
\end{pmatrix}, && 
\overline{\Psi}^C = \begin{pmatrix}
\psi^\alpha & \overline{\chi}_{\dot{\alpha}}
\end{pmatrix}.
\end{align}
The 4-spinor of an arbitrary quark $q$ is given in terms of Weyl spinors $q_L$ and $\overline{q}_R$ by
\begin{align}
q = \begin{pmatrix}
q_L \\
\overline{q}_R
\end{pmatrix}
\end{align}
whereas the 4-spinor of the Dirac gauginos is given in terms of the Weyl spinors $\lambda$ and $\overline{\chi}$\footnote{$\lambda$ is the superpartner of the gluon, called the gluino and $\overline{\chi}$ is the Weyl spinor of the chiral superfield which is associated with the gluon, called the octino.}
\begin{align}
\tilde{g}^a = \begin{pmatrix}
-i \lambda^a \\
i \overline{\chi}^a
\end{pmatrix}
\end{align}

\subsection{Anticommuting numbers}\label{sec:AnticommNumbers}
Anticommuting numbers $\theta^\alpha$ are also referred to as Grassmann numbers and are defined by $\theta^\alpha\theta^\beta = -\theta^\beta\theta^\alpha$ and commute with ordinary numbers.\\
They occur in superspace formalism in the form of 2 tuples, i.e. $\theta^\alpha$ with $\alpha=1,2$. The complex conjugate of this tuple is denoted with $\overline{\theta}^{\dot{\alpha}}$. 
Derivatives are defined by
\begin{align}
\partial^\alpha \theta_\beta := \frac{\partial}{\partial \theta_\alpha} \theta_\beta := \delta^\alpha_\beta \hspace{4cm} \partial_\alpha \theta^\beta := \frac{\partial}{\partial \theta^\alpha} \theta^\beta := \delta_\alpha^\beta\nonumber\\
\overline{\partial}^{\dot{\alpha}} \overline{\theta}_{\dot{\beta}} := \frac{\partial}{\partial \overline{\theta}_{\dot{\alpha}}} \overline{\theta}_{\dot{\beta}} := \delta^{\dot{\alpha}}_{\dot{\beta}} \hspace{4cm} \overline{\partial}_{\dot{\alpha}} \overline{\theta}^{\dot{\beta}} := \frac{\partial}{\partial \overline{\theta}^{\dot{\alpha}}} \overline{\theta}^{\dot{\beta}} := \delta_{\dot{\alpha}}^{\dot{\beta}}
\end{align}
whereby one needs to be cautious as these definitions imply
\begin{align}
\partial_\alpha = -\epsilon_{\alpha\beta}\partial^\beta \hspace{4cm} \partial_{\dot{\alpha}} = -\epsilon_{\dot{\alpha}\dot{\beta}}\partial^{\dot{\beta}}.
\end{align}
Integrals are defined by:
\begin{align}
\int\mathrm{d}\theta_\alpha\ (a+b \theta^\beta + c\theta^\beta \theta^\gamma) &:= b\delta_{\alpha}^{\beta} + c (\delta_\alpha^\beta \theta^\gamma - \delta_\alpha^\gamma \theta^\beta) \hspace{1cm} \mathrm{and} \nonumber\\
\int\mathrm{d}\theta_\alpha\ (a\overline{\theta}^{\dot{\beta}}) &:= (a\overline{\theta}^{\dot{\beta}}) \int\mathrm{d}\theta_\alpha\
\end{align}
where the first line mirrows the claim of translation invariance. One furthermore introduces the shortcuts
\begin{align}
\int \mathrm{d}\theta^2 &:= \int\frac{1}{4}\epsilon_{\alpha\beta} \mathrm{d}\theta^\alpha\mathrm{d}\theta^\beta, \hspace{1cm}
\int \mathrm{d}\overline{\theta}^2 := \int\frac{1}{4}\epsilon_{\dot{\alpha}\dot{\beta}} \mathrm{d}\theta^{\dot{\alpha}}\mathrm{d}\theta^{\dot{\beta}}, \hspace{1cm} \mathrm{and} \nonumber\\
\int \mathrm{d}^4\theta &:= \int \mathrm{d}\theta^2\mathrm{d}\overline{\theta}^2.
\end{align}
For the definition of (anti-)chiral superfields and the field strength chiral superfields it proves useful to introduce supersymmetry (or chiral) covariant derivatives 
\begin{align}
&\mathcal{D}_\alpha := \frac{\partial}{\partial \theta^\alpha} - i(\sigma^\mu\overline{\theta})_\alpha\partial_\mu && \mathcal{D}^{\dot{\alpha}} := \frac{\partial}{\partial \overline{\theta}_{\dot{\alpha}}} - i(\overline{\sigma}^\mu\theta)^{\dot{\alpha}} \partial_\mu\nonumber\\
&D^\alpha := \epsilon^{\alpha\beta} D_\beta = -\frac{\partial}{\partial \theta_\alpha} + i(\overline{\theta} \overline{\sigma}^\mu)^\alpha \partial_\mu && \mathcal{D}_{\dot{\alpha}} := \epsilon_{\dot{\alpha}\dot{\beta}} \mathcal{D}^{\dot{\beta}} = -\frac{\partial}{\partial \overline{\theta}^{\dot{\alpha}}} + i(\sigma^\mu\theta)_{\dot{\alpha}} \partial_\mu.
\end{align}
By construction they are supersymmetry invariant, i.e. they fulfill the following anticommutation relations with the supersymmetry generators from eq. \ref{eq:SUSYGen}:
\begin{align}
\left\{ Q_\alpha, \mathcal{D}_\beta \right\} = \left\{ \overline{Q}_{\dot{\alpha}}, \mathcal{D}_\beta \right\} = \left\{ Q_\alpha, \mathcal{\mathcal{D}}_{\dot{\beta}} \right\} = \left\{ \overline{Q}_{\dot{\alpha}}, \mathcal{\mathcal{D}}_{\dot{\beta}} \right\} = 0
\end{align}


\subsection{Feynman rules for the RSQCD}\label{sec:FeynmanRules}
The following Feynman rules are derived from the Lagrangian of the R-symmetric supersymmetric quantum chromodynamics (RSQCD) eq. \ref{eq:L_RSQCD}. When compared with the Feynman rules of the supersymmetric QCD the diagrams involving scalar gluons are new. In addition the gluon-quark-squark vertex is different in RSQCD for the gauginos are Dirac fermions. Note that for calculations involving fermions one needs to go in the opposite direction of the fermion flow which is given by a curved arrow next to the diagrams in fig. \ref{fig:secondFeynmanRules} and does not always agree with the flow of fermion number given by the direction of the Dirac propagator. This renders the evaluation of fermion number violating processes like $qq \to \tilde{q}\tilde{q}$ unambiguous, cf. \cite{Beenakker:1996ch}.
%refer to paper cited in Beenakker(Appendix: Feynmanrules)
\begin{figure}[!htbp]
\begin{center}
\begin{tikzpicture}[line width=1.5 pt, scale=1.3]
	\node at (-0.5,0.7) {\circled{1}};	
	\draw[scalarnoarrow](0,0)--(1.5,0);
	\node at (180:0.4) {$(\sigma^0)^a$};
	\node at (0:1.9) {$(\sigma^0)^b$};
	\node at (3.4,0) {$\hat{=}\ \frac{i}{p^2-m_{\sigma^0}^2+i\varepsilon}\delta_{ab}$};
\begin{scope}[shift={(7,0)}]
	\node at (-0.5,0.7) {\circled{2}};
	\draw[scalarnoarrow](0,0)--(1.5,0);
	\node at (180:0.4) {$(\phi^0)^a$};
	\node at (0:1.9) {$(\phi^0)^b$};
	\node at (3.4,0) {$\hat{=}\ \frac{i}{p^2-m_{\phi^0}^2+i\varepsilon}\delta_{ab}$};
\end{scope}
\begin{scope}[shift={(0,-2)}]
	\node at (-0.5,0.7) {\circled{3}};
	\draw[scalarbar](0,0)--(1.5,0);
	\node at (180:0.3) {$\tilde{q}_{Ai}$};
	\node at (0:1.8) {$\tilde{q}_{Bj}^\dagger$};
	\node at (3,0) {$\hat{=}\ \frac{i\delta_{AB}}{p^2-m_{\tilde{q}}^2+i\varepsilon}\delta_{ij}$};
\end{scope}
\begin{scope}[shift={(7,-2)}]
	\node at (-0.5,0.7) {\circled{4}};
	\draw[fermionbar](0,0)--(1.5,0);
	\node at (180:0.2) {$q_i$};
	\node at (0:1.7) {$\overline{q}_j$};
	\node at (3,0) {$\hat{=}\ i\frac{\slashed{p}+m_q}{p^2-m_q^2+i\varepsilon}\delta_{ij}$};
\end{scope}
\begin{scope}[shift={(0,-4)}]
	\draw[gluon](1.5,0)--(0,0);
	\node at (-0.5,0.7) {\circled{5}};
	\node at (180:0.2) {$G_\mu^a$};
	\node at (0:1.7) {$G_\nu^b$};
	\node at (3,0) {$\hat{=}\ -i\frac{g^{\mu\nu}}{p^2+i\varepsilon}\delta_{ab}$};
\end{scope}
\begin{scope}[shift={(7,-4)}]
	\node at (-0.5,0.7) {\circled{6}};
	\draw[gluon](1.5,0)--(0,0);
	\draw[fermionnoarrow](0,0)--(1.5,0);
	\node at (180:0.2) {$\tilde{g}^a$};
	\node at (0:1.7) {$\overline{\tilde{g}}^b$};
	\node at (3,0) {$\hat{=}\ i\frac{\slashed{p}+m_{\tilde{g}}}{p^2-m_{\tilde{g}}^2+i\varepsilon}\delta_{ab}$};
\end{scope}
\begin{scope}[shift={(0,-6)}]
	\node at (-0.5,0.7) {\circled{7}};
	\draw[ghostbar](0,0)--(1.5,0);
	\node at (180:0.3) {$c^{a}$};
	\node at (0:1.8) {$\overline{c}^{b}$};
	\node at (3,0) {$\hat{=}\ \frac{i}{p^2+i\varepsilon}\delta_{ab}$};
\end{scope}

\begin{scope}[shift={(1,-8)}]
	\node at (-1.5,0.7) {\circled{8}};
	\draw[fermion] (0,0)--(45:1);
	\draw[fermionbar] (0,0)--(-45:1);
	\draw[gluon] (0,0)--(180:1);
	\node at (45:1.3) {$\overline{q}i$};
	\node at (-45:1.3) {$q_j$};
	\node at (180:1.3) {$G_\mu^a$};
	\node at (2,0) {$\hat{=} -ig_s T^a_{ij}\gamma^\mu$};
\end{scope}
\begin{scope}[shift={(8,-8)}]
	\node at (-1.5,0.7) {\circled{9}};
	\draw[scalar] (0,0)--(45:1);
	\draw[scalarbar] (0,0)--(-45:1);
	\draw[gluon] (0,0)--(180:1);
	\node at (45:1.4) {$\tilde{q}^\dagger_{Ai}(-p_1)$};
	\node at (-45:1.4) {$\tilde{q}_{Bj}(p_2)$};
	\node at (180:1.3) {$G_\mu^a$};
	\node at (3,0) {$\hat{=} -ig_s (p_2+p_1)^\mu T^a_{ij}\delta_{AB}$};
\end{scope}
\begin{scope}[shift={(1,-10.5)}]
	\node at (-1.5,0.7) {\circled{10}};
	\draw[scalar] (0,0)--(45:1);
	\draw[scalarbar] (0,0)--(-45:1);
	\draw[gluon] (0,0)--(135:1);
	\draw[gluon] (0,0)--(-135:1);
	\node at (45:1.4) {$\tilde{q}^\dagger_{Ai}$};
	\node at (-45:1.4) {$\tilde{q}_{Bj}$};
	\node at (135:1.4) {$G_\mu^a$};
	\node at (-135:1.4) {$G_\nu^b$};
	\node at (3,0) {$\hat{=} \ ig_s^2 g^{\mu\nu} \left\{T^a,T^b\right\}_{ij}\delta_{AB}$};
\end{scope}
\begin{scope}[shift={(1,-13.5)}]
	\node at (-1.5,0.7) {\circled{11}};
	\draw[gluon] (0,0)--(45:1);
	\draw[gluon] (0,0)--(-45:1);
	\draw[gluon] (0,0)--(180:1);
	\node at (45:1.4) {$G_\mu^a(p_a)$};
	\node at (-45:1.4) {$G_\nu^b(p_b)$};
	\node at (180:1.5) {$G_\rho^c(p_c)$};
	\node at (5.5,0) {$\hat{=}  - g_s f_{abc} \left[ g_{\mu\nu}(p_a-p_b)^\rho +\  g_{\nu\rho}(p_b-p_c)^\mu +  g_{\rho\mu}(p_c-p_a)^\nu \right]$};
\end{scope}
\end{tikzpicture}
\caption{In the Feynman diagrams of the propagators the momentum is flowing from the right to the left hand side.\newline
In the Feynman diagrams of the vertices all momenta flow towards the vertex, i.e. in diagram 9 $-p_1$ is flowing towards the vertex.\newline
The indices $A,B \in \left\{ L,R \right\}$ label the left/right "handedness" of the squarks. The indices $i,j=1,2,3$ are the color indices in the (anti)fundamental representation where $a,b,c,\hdots = 1,\hdots,8$ are the color indices of the adjoint representation.}\label{fig:firstFeynmanRules}
\end{center}
\end{figure}
\newpage

%The gluino-quark-squark vertex is not much different from the one in SQCD. One only must avoid replacing $\tilde{g}$ with $\tilde{g}^C$. %See for the derivation of Feynman rules for theories with fermion number violating interaction reference [a. denner, h.eck, o. hahn, j. küblbeck]
\begin{figure}[H]
\begin{center}
\begin{tikzpicture}[line width=1.5 pt, scale=1.3]
	\node at (-1.5,0.7) {\circled{12}};
	\draw[gluon] (0,0)--(45:1);
	\draw[gluon] (0,0)--(-45:1);
	\draw[gluon] (0,0)--(135:1);
	\draw[gluon] (0,0)--(-135:1);
	\node at (45:1.4) {$G_\rho^c$};
	\node at (-45:1.4) {$G_\sigma^d$};
	\node at (135:1.4) {$G_\mu^a$};
	\node at (-135:1.4) {$G_\nu^b$};
	\node at (5.5,0) {$\hat{=} \ -ig_s^2 [f^{abe}f^{cde}(g^{\mu\rho} g^{\nu\sigma} - g^{\mu\sigma}g^{\nu\rho}) + f^{ace}f^{bde}(g^{\mu\nu} g^{\rho\sigma} - g^{\mu\sigma}g^{\nu\rho}) $};
	\node at (8,-0.5) {$+ f^{ade}f^{bce}(g^{\mu\nu} g^{\rho\sigma} - g^{\mu\rho}g^{\nu\sigma})]$};
\begin{scope}[shift={(0,-2.5)}]
	\node at (-1.5,0.7) {\circled{13}};
	\draw[gluon] (0,0)--(180:1);
	\draw[gluon] (0,0)--(45:1);
	\draw[gluon] (0,0)--(-45:1);
	\draw[fermionnoarrow] (0,0)--(-45:1);
	\draw[fermionnoarrow] (0,0)--(45:1);
	\node at (45:1.4) {$\overline{\tilde{g}}^b$};
	\node at (-45:1.4) {$\tilde{g}^c$};
	\node at (180:1.5) {$G_\mu^a$};
	\node(gbar) at (0.9,0.6) {};
	\node(g) at (0.9,-0.6) {};
	\path[line width=0.8pt,<-] (g) edge [out=135, in=-135] (gbar);
	\node at (2,0) {$\hat{=}  - g_s f_{abc} \gamma^\mu$};
\end{scope}
\begin{scope}[shift={(7,-2.5)}]
	\node at (-1.5,0.7) {\circled{14}};
	\draw[gluon] (0,0)--(180:1);
	\draw[scalar] (0,0)--(45:1);
	\draw[scalarbar] (0,0)--(-45:1);
	\node at (40:1.4) {$\sigma^{b\dagger}(-p_1)$};
	\node at (-40:1.4) {$\sigma^c(p_2)$};
	\node at (180:1.5) {$G_\mu^a$};
	\node at (3,0) {$\hat{=}  - g_s (p_1+p_2)^\mu f_{abc} $};
\end{scope}
\begin{scope}[shift={(0,-5)}]
	\node at (-1.5,0.7) {\circled{15}};
	\draw[scalar] (0,0)--(45:1);
	\draw[scalarbar] (0,0)--(-45:1);
	\draw[gluon] (0,0)--(135:1);
	\draw[gluon] (0,0)--(-135:1);
	\node at (45:1.4) {$\sigma^{\dagger c}$};
	\node at (-45:1.4) {$\sigma^d$};
	\node at (135:1.4) {$G_\mu^a$};
	\node at (-135:1.4) {$G_\nu^b$};
	\node at (3,0) {$\hat{=} \ +ig_s^2 g^{\mu\nu}[f^{aec}f^{bed}+ f^{bec}f^{aed}]$};
\end{scope}
\begin{scope}[shift={(0,-7.5)}]
	\node at (-1.5,0.7) {\circled{16a}};
	\draw[scalar] (0,0)--(45:1);
	\draw[fermionbar] (0,0)--(-45:1);
	\draw[fermionnoarrow] (180:1)--(0,0);
	\draw[gluon] (180:1)--(0,0);
	\node at (45:1.3) {$\tilde{q}_{Li}^\dagger$};
	\node at (-45:1.3) {$\overline{q}^C_j$};
	\node at (180:1.3) {$\tilde{g}^a$};
	\node(g) at (-0.7,-0.3) {};
	\node(u) at (0.4,-0.8) {};
	\path[line width=0.8pt,<-] (g) edge [out=0, in=135] (u);
	\node at (3,0) {$\hat{=} -i\sqrt{2}g_s T^a_{ij}P_L$};
\end{scope}
\begin{scope}[shift={(7,-7.5)}]
	\node at (-1.5,0.7) {\circled{16b}};
	\draw[fermion] (0,0)--(45:1);
	\draw[scalarbar] (0,0)--(-45:1);
	\draw[fermionnoarrow] (180:1)--(0,0);
	\draw[gluon] (180:1)--(0,0);
	\node at (45:1.3) {$q^C_i$};
	\node at (-45:1.3) {$\tilde{q}_{Lj}$};
	\node at (180:1.3) {$\overline{\tilde{g}}^a$};
	\node(g) at (-0.7,0.3) {};
	\node(u) at (0.4,0.8) {};
	\path[line width=0.8pt,->] (g) edge [out=0, in=225] (u);
	\node at (3,0) {$\hat{=} -i\sqrt{2}g_s T^a_{ij}P_R$};
\end{scope}
\begin{scope}[shift={(0,-10)}]
	\node at (-1.5,0.7) {\circled{17a}};
	\draw[scalarbar] (0,0)--(45:1);
	\draw[fermion] (0,0)--(-45:1);
	\draw[fermionnoarrow] (180:1)--(0,0);
	\draw[gluon] (180:1)--(0,0);
	\node at (45:1.3) {$\tilde{q}_{Rj}$};
	\node at (-45:1.3) {$\overline{q}_i$};
	\node at (180:1.3) {$\tilde{g}^a$};
	\node(g) at (-0.7,-0.3) {};
	\node(u) at (0.4,-0.8) {};
	\path[line width=0.8pt,<-] (g) edge [out=0, in=135] (u);
	\node at (3,0) {$\hat{=} +i\sqrt{2}g_s T^a_{ij}P_L$};
\end{scope}
\begin{scope}[shift={(7,-10)}]
	\node at (-1.5,0.7) {\circled{17b}};
	\draw[fermionbar] (0,0)--(45:1);
	\draw[scalar] (0,0)--(-45:1);
	\draw[fermionnoarrow] (180:1)--(0,0);
	\draw[gluon] (180:1)--(0,0);
	\node at (45:1.3) {$q_j$};
	\node at (-45:1.3) {$\tilde{q}_{Ri}^\dagger$};
	\node at (180:1.3) {$\overline{\tilde{g}}^a$};
	\node(g) at (-0.7,0.3) {};
	\node(u) at (0.4,0.8) {};
	\path[line width=0.8pt,->] (g) edge [out=0, in=225] (u);
	\node at (3,0) {$\hat{=} +i\sqrt{2}g_s T^a_{ij}P_R$};
\end{scope}
\begin{scope}[shift={(0,-12.5)}]
	\node at (-1.5,0.7) {\circled{18a}};
	\draw[scalar] (0,0)--(45:1);
	\draw[fermionnoarrow] (0,0)--(-45:1);
	\draw[gluon] (0,0)--(-45:1);
	\draw[fermionnoarrow] (180:1)--(0,0);
	\draw[gluon] (180:1)--(0,0);
	\node at (45:1.3) {$\sigma^{b\dagger}$};
	\node at (-45:1.3) {$\overline{\tilde{g}}^c$};
	\node at (180:1.3) {$\tilde{g}^a$};
	\node(g) at (-0.7,-0.3) {};
	\node(u) at (0.4,-0.8) {};
	\path[line width=0.8pt,<-] (g) edge [out=0, in=135] (u);
	\node at (3,0) {$\hat{=} -\sqrt{2}g_s f^{abc}P_L$};
\end{scope}
\begin{scope}[shift={(7,-12.5)}]
	\node at (-1.5,0.7) {\circled{18b}};
	\draw[fermionnoarrow] (0,0)--(45:1);
	\draw[gluon] (0,0)--(45:1);
	\draw[scalarbar] (0,0)--(-45:1);
	\draw[fermionnoarrow] (180:1)--(0,0);
	\draw[gluon] (180:1)--(0,0);
	\node at (45:1.3) {$\tilde{g}^b$};
	\node at (-45:1.3) {$\sigma^c$};
	\node at (180:1.3) {$\overline{\tilde{g}}^a$};
	\node(g) at (-0.7,0.3) {};
	\node(u) at (0.4,0.8) {};
	\path[line width=0.8pt,->] (g) edge [out=0, in=225] (u);
	\node at (3,0) {$\hat{=} +\sqrt{2}g_s f^{abc} P_R$};
\end{scope}
\begin{scope}[shift={(0,-15)}]
	\node at (-1.5,0.7) {\circled{19}};
	\draw[scalar] (0,0)--(45:1);
	\draw[scalarbar] (0,0)--(-45:1);
	\draw[scalar] (0,0)--(135:1);
	\draw[scalarbar] (0,0)--(-135:1);
	\node at (45:1.4) {$\tilde{q}^\dagger_{Dl}$};
	\node at (-45:1.4) {$\tilde{q}_{Bj}$};
	\node at (135:1.4) {$\tilde{q}^\dagger_{Ck}$};
	\node at (-135:1.4) {$\tilde{q}_{Ai}$};
	\node at (5,0) {$\hat{=} \ -ig_s^2 [T_{ki}^a T_{lj}^a(\delta_{AL}\delta_{CL} - \delta_{AR}\delta_{CR})(\delta_{BL}\delta_{DL}-\delta_{BR}\delta_{DR})$};
	\node at (5.6,-0.5) {$T_{kj}^a T_{li}^a(\delta_{BL}\delta_{CL} - \delta_{BR}\delta_{CR})(\delta_{AL}\delta_{DL}-\delta_{AR}\delta_{DR})]$};
\end{scope}
\end{tikzpicture}
\end{center}
\caption{The curved arrows indicate the fermion flow. The Feynman rules 16b, 17b and 18b are the complex conjugates of 16a, 17a and 18a respectively. Applying a flipping rule to a vertex one has to reverse the curved arrow, i.e. the fermion flow and replace $\Psi$ with $\bar{\Psi}^C$. In addition one has to add a minus sign for Feynman rule 13.}\label{fig:secondFeynmanRules}
\end{figure}
\begin{figure}[H]
\begin{center}
\begin{tikzpicture}[line width=1.5 pt, scale=1.3]
	\node at (-1.5,0.7) {\circled{20}};
	\draw[scalar] (0,0)--(45:1);
	\draw[scalarbar] (0,0)--(-45:1);
	\draw[scalar] (0,0)--(135:1);
	\draw[scalarbar] (0,0)--(-135:1);
	\node at (45:1.4) {$\sigma^{b\dagger}$};
	\node at (-45:1.4) {$\sigma^c$};
	\node at (135:1.4) {$\tilde{q}^\dagger_{Aj}$};
	\node at (-135:1.4) {$\tilde{q}_{Ai}$};
	\node at (3,0) {$\hat{=} \ -g_s^2 T_{ij}^a f^{abc}(\delta_{AL}\delta_{CL} - \delta_{AR}\delta_{BR})$};
\begin{scope}[shift={(0,-2.5)}]
	\node at (-1.5,0.7) {\circled{21}};
	\draw[scalar] (0,0)--(45:1);
	\draw[scalarbar] (0,0)--(-45:1);
	\draw[scalar] (180:1)--(0,0);
	\node at (45:1.3) {$\tilde{q}^\dagger_{Ai}$};
	\node at (-45:1.3) {$\tilde{q}_{Bj}$};
	\node at (180:1.6) {$\sigma^a+\sigma^{a\dagger}$};
	\node at (3,0) {$\hat{=} -i\sqrt{2}g_s m_{\tilde{g}} T^a_{ij}(\delta_{AL}\delta_{BL}-\delta_{AR}\delta_{BR})$};
\end{scope}
\begin{scope}[shift={(0,-5)}]
	\node at (-1.5,0.7) {\circled{22}};
	\draw[scalar] (0,0)--(45:1);
	\draw[scalarbar] (0,0)--(-45:1);
	\draw[scalar] (0,0)--(135:1);
	\draw[scalarbar] (0,0)--(-135:1);
	\node at (45:1.4) {$\sigma^{\dagger d}$};
	\node at (-45:1.4) {$\sigma^e$};
	\node at (135:1.4) {$\sigma^{\dagger b}$};
	\node at (-135:1.4) {$\sigma^c$};
	\node at (3,0) {$\hat{=} \ -g_s^2 (f^{abc}f^{ade} + f^{abc}f^{adc})$};
\end{scope}
\end{tikzpicture}
\end{center}
\end{figure}


\subsection{Passarino-Veltman Integrals}\label{sec:Passarino}
The definition of the Passarino-Veltman integrals in this thesis agrees with the one from \texttt{Looptools} \cite{Hahn:1998} convention. The original paper \cite{Passarino:1978jh} uses slightly different conventions. A pedagogical introduction to the evaluation of one-loop integrals can be found in \cite{Ellis:2011cr}
\begin{align}
\frac{i}{16\pi^2} A_0(m^2) &= \int_l \frac{1}{l^2-m^2}\nonumber\\
\frac{i}{16\pi^2} B_{0,\mu,\mu\nu}(p^2,m_1^2,m_2^2) &= \int_l \frac{\left\{1,l_\mu,l_\mu l_\nu \right\}}{[l^2-m_1^2][(l+p)^2-m_2^2]}\\
\frac{i}{16\pi^2} C_{0,\mu,\mu\nu}(p_1^2,p_2^2,(p_1+p_2)^2,m_1^2,m_2^2,m_3^2) &= \int_l \frac{\left\{1,l_\mu,l_\mu l_\nu \right\}}{[l^2-m_1^2][(l+p_1)^2-m_2^2][(l+p_1+p_2)^2-m_3^2]}\nonumber
\end{align}
with the shortcut $\int_l = \mu^{2\epsilon}\int\frac{\mathrm{d}^D l}{(2\pi)^D}$. Furthermore there are supressed $\varepsilon$'s which prescripe how the poles in the complex plane are avoided. They are hidden in the infinitesimal shift of the masses: $m_i^2 \to m_i^2 - i \varepsilon$.\\
The tensor integrals can be decomposed as
\begin{align}
B_\mu &:= p_\mu B_1\nonumber\\
B_{\mu\nu} &:= g_{\mu\nu}B_{00} + p_\mu p_\nu B_{11}\nonumber\\
C_\mu &= p_{1\mu}C_1 + p_{2\mu}C_2\\
C_{\mu\nu} &:= g_{\mu\nu}C_{00} + p_{1\mu}p_{1\nu}C_{11} + p_{2\mu}p_{2\nu}C_{22} + (p_{1\mu}p_{2\nu} + p_{2\mu}p_{1\nu})C_{12}\nonumber
\end{align}
In the the special case of vanishing momenta the integrals take a succinct form. %The following formulae show some of these special cases.
\begin{align}
A_0(m^2) = m^2\left( \Delta_\epsilon -\ln \frac{m^2}{\mu^2} + 1 \right) + \mathcal{O}(\epsilon)
\end{align}
where the typical UV-divergent constant $\Delta_\epsilon = \frac{1}{\epsilon} - \gamma_E+\ln 4\pi$ is defined. It comprises the Euler-Mascheroni constant $\gamma_{\mathrm{E}}$.
\begin{align}
B_0(0,m_1^2,m_2^2) &= \frac{A_0(m_1^2)-A_0(m_2^2)}{m_1^2-m_2^2} = \Delta_\epsilon + 1 -\frac{m_1^2\ln \frac{m_1^2}{\mu^2}-m_2^2\ln \frac{m_2^2}{\mu^2}}{m_1^2-m_2^2} + \mathcal{O}(\epsilon)\\
B_0(0,m^2,m^2) &= \frac{\partial}{\partial m^2} A_0(m^2) = \Delta_\epsilon -\ln \frac{m^2}{\mu^2} + \mathcal{O}(\epsilon)\\
B_0(0,0,0) &= 0\\
B_1(0,m_1^2,m_2^2) &= -\frac{\Delta_\epsilon}{2} + \frac{1}{2}\ln \frac{m_1^2}{\mu^2} + \frac{-3+4t-t^2-4t \ln t +2t^2 \ln t}{4(t-1)^2} + \mathcal{O}(\epsilon)\label{eq:B1}\\
B_1(0,0,0) &= 0
\end{align}
The scaleless integrals are defined to be zero due to their definition  on how they scale in $D$ dimensions\cite{Collins:105730}. The parameter $t$ is given by $\frac{m_2^2}{m_1^2}$. As can be seen from \ref{eq:B1} $B_1$ is in contrast to $B_0$ not symmetric in its masses but it can be shown [Romao] that
\begin{align}
B_1(p^2,m_1^2,m_2^2) = -(B_0(p^2,m_2^2,m_1^2) + B_1(p^2,m_2^2,m_1^2))
\end{align}
It can further be shown that
\begin{align}
C_{00}(0,0,0,m_1^2, m_1^2, m_2^2) = -\frac{1}{2}B_1(0, m_1^2, m_2^2)
\end{align} and that $C_{00}(0,0,0,m_1^2, m_1^2, m_2^2)$ is a symmetric function of its masses.\\
From the generic $\epsilon$-expansion of the $B_0$ integral \cite{Denner}
\begin{align}
B_0(p^2,m_1^2,m_2^2) = \Delta_\epsilon - \int_0^1\mathrm{d}x\ \ln \frac{-x(1-x)p^2 + xm_2^2 + (1-x)m_1^2}{\mu^2}
\end{align} 
and Passarino-Veltman decomposition \cite{Passarino:1978jh} one can determine the UV-divergent part of all $B$ and $C$ integrals. In chapter \ref{sec:MSRen} this was necessary in order to obtain the renormalization constants in the $\overline{\mathrm{MS}}$-scheme. Infrared and collinear singularities arise from the special case where one or multiple masses tend to zero. These poles are either regularized in terms of a small mass cutoff $\Lambda$ or also dimensionally as $\epsilon$-poles. In the later case the integral first needs to undergo the limit to zero masses and than being evaluated.\\
The following list shows all necessary integrals needed to determine the renormalization constants in \ref{sec:MSRen}.
\begin{align}
\left.A_0(m^2)\right|_{\mathrm{UV-div}} &= m^2 \Delta_\epsilon\\
\left.B_0(p^2,m_1^2,m_2^2)\right|_{\mathrm{UV-div}} &= \Delta_\epsilon\\
\left.B_1(p^2,m_1^2,m_2^2)\right|_{\mathrm{UV-div}} &= -\frac{1}{2}\Delta_\epsilon\\
\left.C_{00}(p_1^2,p_2^2,(p_1+p_2)^2,m_1^2,m_2^2,m_3^2)\right|_{\mathrm{UV-div}} &= \frac{1}{4}\Delta_\epsilon
\end{align}
The UV-divergent part of $C_{11}$, $C_{22}$, $C_{12}$ equals zero. As can be seen already from the superficial degree of divergence also $C_i|_{\mathrm{UV-div}} = 0$ for $i \in \left\{ 0, 1, 2 \right\}$ .



\subsection{Cross section and Phase Space Integration}\label{sec:cross_section}
Once the Feynman amplitude $\mathcal{M}$ for a $2 \to N$ body scattering\footnote{with kinematics $k_a + k_b \to p_1 + \hdots p_N$} is computed one can calculate physical observables with it. The differential cross section for $2 \to N$ scattering is given by
\begin{align}
\mbox{d}\sigma = \frac{1}{F}\mathrm{d}\Phi_N |\mathcal{M}|^2.
\end{align}
The flux factor is defined by $F = 4\sqrt{(k_a\cdot k_b)^2 - (m_am_b)^2}$ which equals $F = 2s$ for massless initial state particles. The differential for the $N$ body phase space in $D$ dimensions is given by
\begin{align}
\mathrm{d}\Phi_N = \left( \mu^{2\epsilon} \right)^{N-1} \left( \prod_{f=1}^N \frac{\mbox{d}^{D-1}p_f}{(2\pi)^{D-1}}\frac{1}{2E_f} \right)  (2\pi)^D \delta^{(D)}(k_a+k_b-\sum_{f=1}^N p_f).
\end{align} 
The factor $\mu^{2\epsilon}$ is included to maintain the mass dimension of the cross section. As in this thesis the sum of $|\mathcal{M}|^2$ over helicities and colors $\sum |\mathcal{M}|^2$ has been calculated one can write
\begin{align}
\mathrm{d}\sigma = \frac{1}{2s} \mathrm{d}\Phi_2 K_{ab} \sum |\mathcal{M}|^2
\end{align} 
where $K_{ab}$ encodes the averaging over initial state helicities and colors. Specifying to the center-of-mass frame and assuming that $\sum|\mathcal{M}|^2$ is only a function of the modulus of one of the final state particle's 3-momentum $ |\vec{p}_i|$ and the angle $\theta$ between $\vec{k}_a$ and $\vec{p}_1$ one can write
\begin{align}
\int \mathrm{d}\Phi_2 &= \mu^{2\epsilon} \int \frac{\mathrm{d}|\vec{p}_1|\mathrm{d}\Omega^{D-1}_1}{(2\pi)^{D-2} 4 E_1 E_2} |\vec{p}_1|^{D-2} \delta \left(k_a^0 + k_b^0 - \sqrt{m_1^2 + |\vec{p}_1|^2} - \sqrt{m_2^2 + |\vec{p}_1|^2} \right)\nonumber\\
&= \frac{1}{(2\pi)^{D-2}}\frac{2\pi^{\frac{D}{2}-1}}{\Gamma(\frac{D}{2}-1)} \mu^{2\epsilon} \int_0^\infty\mathrm{d}|\vec{p}_1| \int_0^\pi \mathrm{d} \cos\theta \frac{1}{4 E_1 E_2} p_1^{D-2}\sin^{D-4}\theta\nonumber\\
&\ \ \delta \left(k_a^0 + k_b^0 - \sqrt{m_1^2 + |\vec{p}_1|^2} - \sqrt{m_2^2 + |\vec{p}_1|^2} \right).\label{eq:DPhi2}
\end{align}
In the second line the integral over the D-dimensional hypersphere 
\begin{align}
\int \mathrm{d}\Omega^D = \int_0^{2\pi} \mathrm{d}\phi \prod_{i=1}^{D-2}\int_0^\pi \sin^i\theta_i \mathrm{d}\theta_i = \frac{2\pi^{\frac{D}{2}}}{\Gamma(\frac{D}{2})}
\end{align}
has been used. Because $\sum|\mathcal{M}|^2$ is calculated in terms of Mandelstam variables 
\begin{align}
t &= (k_a-p_1)^2  \nonumber\\
t &= -2\left(|\vec{k}_a| \sqrt{m_1^2 + |\vec{p}_1|^2} - |\vec{k}_a||\vec{p}_1| \cos\theta\right) + m_1^2\label{eq:t} \\
u &= (k_a-p_2)^2\nonumber\\
u &= -2\left(|\vec{k}_a| \sqrt{m_2^2 + |\vec{p}_1|^2} + |\vec{k}_a||\vec{p}_1| \cos\theta\right) + m_2^2\label{eq:u}
\end{align}
it is useful to perform a change of coordinates yielding
\begin{align}
\mathrm{d}|\vec{p}_1|\ \mathrm{d}\cos\theta = -\frac{E_1 E_2}{4|\vec{k}_a|^2|\vec{p}_1|^2(E_1+E_2)}\mathrm{d}u\ \mathrm{d}t.\label{eq:PhaseTrafo}
\end{align}
Inserting \ref{eq:PhaseTrafo} into \ref{eq:DPhi2} and using $2|\vec{k}_a| = \sqrt{s} = E_1 + E_2$ gives
\begin{align}
\int \mathrm{d}\Phi_2 = & \frac{1}{s} \frac{\pi^{-\frac{D}{2}+1}}{2^{D-3}\Gamma(\frac{D}{2}-1)} \int \mathrm{d}u\ \mathrm{d}t\ \left( \frac{tu-m_1^2m_2^2}{\mu^{2\epsilon} s} \right)^{\frac{D-4}{2}} \nonumber\\
&\frac{1}{4}\Theta(tu-4m_1^2m_2^2)\delta \left(s+t+u-m_1^2-m_2^2\right)
\end{align}
where the $\Theta$-function comes from the bounds of $|\vec{p}_1|$ and $\theta$ visible in \ref{eq:DPhi2} and the combination of \ref{eq:t} and \ref{eq:u}. Working in $D=4-2\epsilon$ dimensions and inserting $\Theta(s-4m^2)$ with $m = \frac{m_1 + m_2}{2}$ to account for the production threshold one finds
\begin{align}
\frac{\mbox{d}^2 \sigma}{\mbox{d}t\mbox{d}u} =& \frac{K_{ab}}{s^2} \frac{\pi S_{\epsilon}}{\Gamma(1-\epsilon)} \left[ \frac{tu-m_1^2m_2^2}{\mu^2 s}\right]^{-\epsilon} \Theta(tu-m_1^2m_2^2)\nonumber\\
&\ \Theta(s-4m^2) \delta(s+t+u-m_1^2-m_2^2) \sum |\mathcal{M}|^2
\end{align}
where $S_\epsilon = (4\pi)^{-2+\epsilon}$ as defined in \cite{Beenakker:1996ch}.
The averaging factors $K_{ab}$ are given by
\begin{align}
K_{qq} = \frac{1}{4N_c^2} \hspace{1.5cm}K_{GG} = \frac{1}{4(1-\epsilon)^2(N_c^2-1)^2} \hspace{1.5cm}K_{qG} = \frac{1}{4(1-\epsilon)N_c(N_c^2-1)}.
\end{align}

