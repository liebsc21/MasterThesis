\documentclass[
 paper=A4,pagesize=automedia,fontsize=12pt,
 BCOR=15mm,DIV=22,
 twoside,headinclude,footinclude=false,
 ngerman,%fleqn,            % fleqn = linksbündige Ausrichtung von Formeln
 bibtotocnumbered,          % Literaturverz. im Inhaltsverz. eintragen
 liststotoc,                % Abbildungsverz. im Inhaltsverz. eintragen
 listsleft,                 % Abbildungsverz. an der längsten Nummer ausrichten
 pointlessnumbers,          % kein Punkt nach Überschriftsnummerierung
 cleardoublepage=empty      % Vakatseiten ohne Paginierung
]{scrartcl}
\setlength\parindent{0em}

% Kodierung, Schrift und Sprache auswählen
\usepackage[utf8]{inputenc}
\usepackage[T1]{fontenc}
\usepackage[english]{babel}


%% damit man Text aus dem PDF korrekt rauskopieren kann
%\usepackage{cmap}
% Layout: Kopf-/Fußzeilen, anderthalbfacher Zeilenabstand
\usepackage{scrpage2} \pagestyle{scrheadings}
                      \clearscrheadfoot
                      \ihead{\headmark}\ohead{\pagemark}
                      \automark[subsection]{section}
                      \setheadsepline{0.5pt}
\usepackage{setspace} \onehalfspacing
%\deffootnote{1em}{1em}{\textsuperscript{\thefootnotemark }}
% Grafiken, Tabellen, Mathematikumgebungen
\usepackage{graphicx,xcolor}
\usepackage{tabularx}
\usepackage{amsmath,amsfonts,amssymb}
% Darstellung von Fließumgebungen
\usepackage{flafter,afterpage}
\usepackage[section]{placeins}
\usepackage[margin=8mm,font=small,labelfont=bf,format=plain]{caption}
\usepackage[margin=8mm,font=small,labelfont=bf,format=plain]{subcaption}

\numberwithin{equation}{section}
\numberwithin{figure}{section}
\numberwithin{table}{section}

%%%%%%%%%%%%%%%%%%%%%%%%%%%%%%%%%%%%%%%%%%%%%%%%%%%%%%%%%%%%%%%%%%%%%%%%%%%%%%%%
% Ab hier ist Platz für eigene Ergänzungen (Pakete, Befehle, etc.)

% hyperlink table of contents
\usepackage[colorlinks=true,linkcolor=black]{hyperref}
% double horizontal lines in tables
\usepackage{hhline}
% fat vertical lines for tabulars
\usepackage{array}
\newcolumntype{?}{!{\vrule width 2pt}} 
% fat vertical lines for tabulars
\makeatletter
\def\hlinewd#1{%
\noalign{\ifnum0=`}\fi\hrule \@height #1 %
\futurelet\reserved@a\@xhline}
\makeatother 
% to assign tables and figures in the right place
\usepackage{float}
% Referenzen fürs Zitiern
\usepackage{cite}         
% Stil des Quellenverzeichnis     
\bibliographystyle{ieeetr}    
% Package for nice seperation of number and unit
\usepackage[]{units}
% Package for code extracts
\usepackage{listings}
\lstset{breaklines=true,
  breakatwhitespace=true,
  stepnumber=1,
  basicstyle=\ttfamily\footnotesize,
  commentstyle=\ttfamily,
  prebreak={\textbackslash},
  breakindent=10pt,
  breakautoindent=false,
  showspaces=false,
  showstringspaces=false,
  frame=single,
  abovecaptionskip=0em,
  aboveskip=1.4em,
  belowcaptionskip=0.5em,
  belowskip=1em,
  keywordstyle=\ttfamily,
}
\definecolor{light}{RGB}{245,245,245}
\lstdefinestyle{Mybash}{
  language=bash,
  backgroundcolor=\color{light}
}
\lstdefinestyle{MyMathematica}{
  language=Mathematica,
  backgroundcolor=\color{light}
}
\lstdefinestyle{Mycpp}{
  language=C++,
  backgroundcolor=\color{light}
}
% diagonal separtion in table
\usepackage{slashbox}

% comment within a line
\newcommand{\ignore}[1]{}

