\section{Renormalization of the MRSSM}
In order to improve the prediction of the cross section of the previous chapter one has to take quantum corrections into account. These are associated with loops in the corresponding Feynman diagrams. Computing these loop diagrams one might encounter infinities which arise from certain momentum configurations of the unspecified loop momentum. These infinities can be classified due to their origin. Infinities which are associated with loop momenta which tend to infinity are referred to as ultraviolet(UV) divergences. Infinities arising from loop momenta approaching zero can occur in loops with massless particles and are called infrared(IR) singularities. Furthermore there are collinear singularities with occur when a massless particle splits into two massless collinear particles.\\
These infinities are not physical and must therefore be removed to get sensible predictions. To this end one regularizes them to extract them from the quantity in question. UV-divergences can be removed by means of renormalization, i.e. counterterms are inserted into the Lagrangian to cancel UV-divergences. Infrared and collinear divergenves are removed by adding up all possible contributions which give rise to the considered observable.

\subsection{Regularization Schemes}\label{sec:RegSchemes}
\subsubsection*{Dimensional Regularization(DREG)}
Dimensional regularization(DREG) is a very common procedure for regularizing infinities which was devised by t'Hooft and Veltman~\cite{'tHooft:1972fi}. In this scheme loop momenta, gamma- and epsilon-tensors, phase space and fields are defined in $D$ dimensions. As in every regularization scheme a parameter with mass dimension needs to be introduced. In DREG that is the $\mu$ parameter which ensures that the loop integrals still have mass dimension 4:
\begin{align}
\int \frac{\mbox{d}^4p}{(2\pi)^4} \to \mu^{4-D} \int \frac{\mbox{d}^Dp}{(2\pi)^D}.
\end{align}
One often writes $D=4-2\epsilon$. Then the divergences of the loop integral manifest in $\frac{1}{\epsilon}$ poles.\\
However DREG suffers a flaw in supersymmetry. As the degrees of freedom for a massless gauge boson are $D-2$ but the degrees of freedom for its superpartner are 2 there is a mismatch if $D \neq 4$. As a consequence there are $2\epsilon$ degrees of freedom associated with the gluon\footnote{These degrees of freedom are identified with scalars and are therefore referred to as $\epsilon$ scalars} which do not have a supersymmetric partner. Therefore DREG violates supersymmetry.

\subsubsection*{Dimensional Reduction (DRED)}
Dimensional reduction (DRED) was introduced to rectify the imperfections of DREG, i.e. it preserves supersymmetry\footnote{It is not clear if DRED preserves supersymmetry at all orders in perturbation theory but it does preserve supersymmetry at the 1-loop level.}. DRED promotes only loop momenta to $D$ dimensions. All other quantities which are $D$ dimensional in DREG stay in 4 dimensions.\\
maybe refer to Collins: Renormalization


\subsection{Regularization Scheme Dependences}\label{sec:RegSchemeDep}
It is useful to introduce the effective action $\Gamma$ to discuss the subject of this and ensuing subchapters. A formal introduction of $\Gamma$ can be found in~\cite{Peskin}. In short $\Gamma$ can be viewed as a modification of the classical action $\Gamma_{cl} = \int \mathcal{L}_{cl}$ by quantum effects:
\begin{align}
\Gamma = \Gamma_{cl} + \mathcal{O}(\hbar)
\end{align}
This means that in addition to the vertices in the classical Lagrangian new vertices arise due to loop effects. As already suggested loop corrections might a priori not be finite and then need to be made finite by the addition of counterterms. For $\mathcal{O}(\hbar)$ corrections one writes
\begin{align}
\Gamma^{(\leq 1)} \to \Gamma^{(\leq 1)} + \Gamma^{(1),ct}
\end{align} 
These counterterms depend on the regularization (and renormalization) scheme. If one chooses to work with DREG supersymmetry will not be preserved at 1-loop order, i.e. $\Gamma^{(\leq 1)}_{DREG}$ is not supersymmetric. To maintain supersymmetry invariance of the renormalized effective action the counterterms will not only consist of supersymmetric counterterms $\Gamma^{(1),ct,sym}_{DREG}$ but also of counterterms restoring supersymmetry $\Gamma^{(1),ct,restore}_{DREG}$. 
\begin{align}
\Gamma^{(1),ct}_{DREG} = \Gamma^{(1),ct,sym}_{DREG} + \Gamma^{(1),ct,restore}_{DREG}
\end{align}
Fortunately a supersymmetry conserving regularization scheme (at 1-loop level) is given by DRED \cite{Hollik:2001cz}. One way to acquire supersymmetry restoring counterterms is therefore given by
\begin{align}
\Gamma^{(\leq 1)}_{DRED} + \Gamma^{(1),ct}_{DRED} \overset{!}{=} \Gamma^{(\leq 1)}_{DREG} + \Gamma^{(1),ct}_{DREG}.
\end{align}
Setting also the finite terms in $\Gamma^{(1),ct,sym}$ equal in DRED and DREG the choice of the supersymmetry restoring counterterms is fixed by:
\begin{align}
\Gamma^{(1),ct,restore}_{DREG} = \Gamma^{(\leq 1)}_{DRED} - \Gamma^{(\leq 1)}_{DREG}.\label{eq:GammaCtRestore}
\end{align} 
This way supersymmetry is preserved by the renormalization constants.\\ 
In the case of the MRSSM it will turn out that the only supersymmetry violation comes from correction associated with the gluon as already alluded to in \ref{sec:RegSchemes}. However supersymmetry restoring is always already included in $\delta Z^{\mathrm{DREG}}$. This is $\delta Z^{\mathrm{DREG}} = \delta Z^{\mathrm{DREG,sym}} + \delta Z^{\mathrm{trans}}$ where $\delta Z^{\mathrm{trans}} = \delta Z^{\mathrm{DREG}} - \delta Z^{\mathrm{DRED}}$ is the supersymmetry restoring renormalization constant. The only point where particularly care is required is the coupling: The  gauge coupling $g_s$ and the Yukawa coupling $\hat{g}_s$ receive different supersymmetry restoring counterterms. Therefore one has to make a difference between these couplings at 1-loop level. In order to match $g_s$ to the experimentally measured coupling it is renormalized in $\overline{\mathrm{MS}}$. The Yukawa coupling $\hat{g}_s$ therefore needs to be added with the difference of the supersymmetry restoring counterterms at 1-loop in order to be renormalized the same way.


%An alternative procedure is used in \cite{Beenakker:1996ch}, where only one supersymmetry restoring counterterm, which translates between the renormalized gauge coupling $g_s$ and the renormalized Yukawa coupling $\hat{G}_s$ is needed. That is possible because the only supersymmetry violation comes from the gauge sector. For a comprehensive discussion of this issue see [Philipp Vasros Diplomarbeit and the paper with Dominik]\\
%SUSY restoring is always already in $\delta Z^{DREG}$ included. Only important thing is the coupling: $g_s$ and $\hat{g}_s$ receive different SUSY restoring counterterm. Therefore one has to make a difference between these couplings at 1-loop level. In order to match $g_s$ to the experimentaly measured coupling it is renormalized in $\overline{\mathrm{MS}}$. The Yukawa coupling $\hat{g}_s$ therefore needs to be added with the difference of the SUSY-restoring counterterms at 1-loop in order to be renormalized the same way.


\subsection{On-Shell Renormalization}\label{sec:QuarkSE}
A part of the computation of NLO processes is the calculation of renormalization constants. The field and mass renormalization constants have been calculated in DREG in the on-shell scheme. This has the advantaged that when turning to the cross section no manipulation of the Green function to the S-matrix element has to be done. 

\subsubsection{The Quark Self-Energy}
The quark self-energy splits into contribution from the SM as well as a supersymmetric analogue which is already present in the MSSM.
\begin{figure}[!htbp]
\begin{center}
\begin{tikzpicture}[line width=1.5 pt, scale=1.3]
	\draw[fermionbar](180:1.3)--(180:0.5);
	\draw[fermionnoarrow] (-0.5,0) arc (180:0:.5);
	\draw[fermionnoarrow] (0.5,0) arc (0:-180:.5);	
	\node at (0,0) {1L};
	\draw[fermionbar](0:0.5)--(0:1.3);
	\node at (1.8,0) {=};
\begin{scope}[shift={(3.5,0)}]
	\draw[fermionbar](180:1.3)--(180:0.5);
	\draw[gluon] (-0.5,0) arc (180:0:.5);
	\node at (0,0.8) {$G$};
	\draw[fermion] (0.5,0) arc (0:-180:.5);
	\node at (0,-0.8) {$q$};	
	\draw[fermionbar](0:0.5)--(0:1.3);
	\node at (1.8,0) {+};
\end{scope}
\begin{scope}[shift={(7,0)}]
	\draw[fermionbar](180:1.3)--(180:0.5);
	\draw[gluon] (-0.5,0) arc (180:0:.5);
	\draw[fermionnoarrow] (-0.5,0) arc (180:0:.5);
	\node at (0,0.8) {$\tilde{g}$};
	\draw[scalar] (0.5,0) arc (0:-180:.5);
	\node at (0,-0.8) {$\tilde{q}$};	
	\draw[fermionbar](0:0.5)--(0:1.3);
\end{scope}
\end{tikzpicture}
\caption{diagrammatic contributions to the self-energy of the quark at 1-loop level}\label{fig:QuarkSelfEnergy}
\end{center}
\end{figure}\\
The 1-PI diagrams evaluate to
\begin{align}
i\Gamma^{1L}_{q_i\overline{q}_j} = i \frac{g_s^2}{16\pi^2}\delta_{ij} C(F) \left[ 2 \left( B_0(p^2,0,0) + B_1(p^2,0,0)-\frac{1}{2}\right)\slashed{p} - 2 B_1(p^2,m_{\tilde{g}}^2,m_{\tilde{q}}^2)\slashed{p}\right].
\end{align}
With the counterterm Feynman rule\\
\\
\begin{tikzpicture}[line width=1.0 pt, scale=0.8]
	\node at (-2.5,-0.1) {$i\Gamma^{1L,ct}_{q_i\overline{q}_j}\ \hat{=}\ i$};
\begin{scope}[shift={(0.1,0)}]	
	\draw[fermionbar] (-1.3,0) --(0,0);
	\draw[fermionnoarrow] (-0.3,0.3) -- (0.3,-0.3);
	\draw[fermionnoarrow] (-0.3,-0.3) -- (0.3,0.3);
	\draw[fermionbar] (0,0) --(1.3,0);
	\node at (1.6,0) {$j$};
	\node at (3,0) {$\hat{=}\ i\delta_{ij}\delta Z_q \slashed{p} $};	
\end{scope}
\end{tikzpicture}\\
and the on-shell renormalization condition
\begin{align}
\frac{\partial}{\partial \slashed{p}} \left[ \Re (\Gamma^{\mathrm{1L}}_{q_i\overline{q}_j}) + \Gamma^{\mathrm{1L,ct}}_{q_i\overline{q}_j}  \right]_{p^2 = 0} = 0
\end{align}
where $\Re (\hdots)$ denotes the real part of $\hdots$ one finds
\begin{align}
\delta Z_q = 2 C(F) \frac{g_s^2}{16\pi^2} \Re \left[ B_1(p^2,m_{\tilde{g}}^2,m_{\tilde{q}}^2) + \frac{1}{2} \right].
\end{align}
Doing the same calculation in DRED one finds that the second term in the squared brackets is absent. Therefore the transition counterterm between DREG and DRED is given by
\begin{align}
\delta Z_q^{\mathrm{trans}} = \delta Z_q^{\mathrm{DREG}} - \delta Z_q^{\mathrm{DRED}} = C(F) \frac{g_s^2}{16\pi^2}.\label{eq:QuarkSC}
\end{align}

\subsubsection{The Squark Self-Energy}
The contributions to the self-energy of the left- and right-handed squark are the same. Therefore to avoid unnecessary labeling $\Gamma_{\tilde{q}\tilde{q}^\dagger}$ stands in the following for $\Gamma_{\tilde{q}_L\tilde{q}_L^\dagger} = \Gamma_{\tilde{q}_R\tilde{q}_R^\dagger}$.
\begin{figure}[!htbp]
\begin{center}
\begin{tikzpicture}[line width=1.5 pt, scale=1.3]
	\draw[scalarbar](180:1.3)--(180:0.5);
	\draw[fermionnoarrow] (-0.5,0) arc (180:0:.5);
	\draw[fermionnoarrow] (0.5,0) arc (0:-180:.5);	
	\node at (0,0) {1L};
	\draw[scalarbar](0:0.5)--(0:1.3);
	\node at (1.8,0) {=};
\begin{scope}[shift={(3.5,0)}]
	\draw[scalarbar](-1.3,0)--(0,0);
	\draw[scalarbar](0,0)--(1.3,0);
	\draw[scalar] (0,0) arc (-90:270:.5);
	\node at (0,1.3) {$\tilde{q}$};
	\node at (1.8,0) {+};
\end{scope}
\begin{scope}[shift={(7.0,0)}]
	\draw[scalarbar](-1.3,0)--(0,0);
	\draw[scalarbar](0,0)--(1.3,0);
	\draw[gluon] (0,0) arc (270:-90:.5);
	\node at (0,1.3) {$G$};
	\node at (1.8,0) {+};
\end{scope}
\begin{scope}[shift={(10.5,0)}]
	\draw[scalarbar](180:1.3)--(180:0.5);
	\draw[gluon] (-0.5,0) arc (180:0:.5);
	\draw[fermionnoarrow] (-0.5,0) arc (180:0:.5);
	\node at (0,0.8) {$\tilde{g}$};
	\draw[fermion] (0.5,0) arc (0:-180:.5);
	\node at (0,-0.8) {$q$};	
	\draw[scalarbar](0:0.5)--(0:1.3);
\end{scope}
\begin{scope}[shift={(3.5,-2.5)}]
	\node at (-1.8,0) {+};
	\draw[scalarbar](180:1.3)--(180:0.5);
	\draw[scalarnoarrow] (-0.5,0) arc (180:0:.5);
	\node at (0,0.8) {$\phi^0,\ \sigma^0$};
	\draw[scalar] (0.5,0) arc (0:-180:.5);
	\node at (0,-0.8) {$\tilde{q}$};	
	\draw[scalarbar](0:0.5)--(0:1.3);
	\node at (1.8,0) {+};
\end{scope}
\begin{scope}[shift={(7,-2.5)}]
	\draw[scalarbar](180:1.3)--(180:0.5);
	\draw[gluon] (-0.5,0) arc (180:0:.5);
	\node at (0,0.8) {$G$};
	\draw[scalar] (0.5,0) arc (0:-180:.5);
	\node at (0,-0.8) {$\tilde{q}$};	
	\draw[scalarbar](0:0.5)--(0:1.3);
\end{scope}
\end{tikzpicture}
\caption{diagrammatic contributions to the self-energy of the squark at 1-loop level}
\end{center}
\end{figure}\\
\begin{align}
i\Gamma^{1L}_{\tilde{q}_{i}\tilde{q}^\dagger_j} &= i \frac{g_s^2}{16\pi^2}\delta_{ij}C(F) \left[  A_0(m_{\tilde{q}}^2) + 0 - \left( 4A_0(m_{\tilde{g}}^2) + 4B_1(p^2,0,m_{\tilde{g}}^2)p^2 \right) \right.\nonumber\\
& \left.+ 4m_{\tilde{g}}^2B_0(p^2,m_{\phi^0}^2,m_{\tilde{q}}^2) - \left(2B_1(p^2,0,m_{\tilde{q}}^2)p^2 + B_0(p^2,0,m_{\tilde{q}}^2)(m_{\tilde{q}}^2+3p^2) \right) \right].
\end{align}
Suppressing $\delta_{AB}$ with $A,B = L,R$ which is present in the tree level propagator the counterterm Feynman rule is given by\\
\\
\begin{tikzpicture}[line width=1.0 pt, scale=0.8]
	\node at (-2.5,-0.1) {$i\Gamma^{1L,ct}_{q_i\overline{q}_j}\ \hat{=}\ i$};
\begin{scope}[shift={(0.1,0)}]	
	\draw[scalarbar] (-1.3,0) --(0,0);
	\draw[fermionnoarrow] (-0.3,0.3) -- (0.3,-0.3);
	\draw[fermionnoarrow] (-0.3,-0.3) -- (0.3,0.3);
	\draw[scalarbar] (0,0) --(1.3,0);
	\node at (1.6,0) {$j$};
	\node at (5,0) {$\hat{=}\ i\delta_{ij} \left[ \delta Z_{\tilde{q}}(p^2-m_{\tilde{q}}^2) - \delta m_{\tilde{q}}^2 \right] .$};	
\end{scope}
\end{tikzpicture}\\
The on-shell renormalization conditions read
\begin{align}
&\frac{\partial}{\partial p^2} \left[ \Re (\Gamma^{\mathrm{1L}}_{\tilde{q}_{i}\tilde{q}^\dagger_j}) + \Gamma^{\mathrm{1L,ct}}_{\tilde{q}_{i}\tilde{q}^\dagger_j}  \right]_{p^2 = m_{\tilde{q}}^2} = 0 && \left[ \Re (\Gamma^{\mathrm{1L}}_{\tilde{q}_{i}\tilde{q}^\dagger_j}) + \Gamma^{\mathrm{1L,ct}}_{\tilde{q}_{i}\tilde{q}^\dagger_j}  \right]_{p^2 = m_{\tilde{q}}^2} = 0.
\end{align}
This results in the following renormalization constants
\begin{align}
\delta Z_{\tilde{q}} &= \frac{g_s^2}{16\pi^2}C(F) \Re \left[ 4B_1(p^2,0,m_{\tilde{g}}^2) + 2B_1(B_1(p^2,0,m_{\tilde{q}}^2)) + 3B_0(p^2,0,m_{\tilde{q}}^2) \right.\nonumber\\
& + 4m_{\tilde{q}}^2 \frac{\partial}{\partial p^2}B_1(p^2,0,m_{\tilde{g}}^2) - 4m_{\tilde{g}}^2\frac{\partial}{\partial p^2}B_0(p^2,m_{\phi^0}^2,m_{\tilde{q}}^2) + 2m_{\tilde{q}}^2 \frac{\partial}{\partial p^2}B_1(p^2,0,m_{\tilde{q}}^2)\nonumber\\
& \left. + 4m_{\tilde{q}}^2 \frac{\partial}{\partial p^2}B_0(p^2,0,m_{\tilde{q}}^2)\right]_{p^2 = m_{\tilde{q}}^2}
\end{align}
\begin{align}
\delta m_{\tilde{q}}^2 = \frac{g_s^2}{16\pi^2}C(F)&\left[ A_0(m_{\tilde{q}}^2) - (4A_0(m_{\tilde{g}}^2) + 4B_1(m_{\tilde{q}}^2,0,m_{\tilde{g}}^2)m_{\tilde{q}}^2) + 4 m_{\tilde{g}}^2 B_0(m_{\tilde{q}}^2,m_{\phi^0}^2,m_{\tilde{q}}^2) \right.\nonumber\\
&\left. -(2B_1(m_{\tilde{q}}^2,0,m_{\tilde{q}}^2) m_{\tilde{q}}^2 + 4 B_0(m_{\tilde{q}}^2,0,m_{\tilde{q}}^2)m_{\tilde{q}}^2) \right]
\end{align}
The squark self-energy exhibits no regularization dependence. The transition counterterms are therefore 
\begin{align}
\delta Z_{\tilde{q}}^{\mathrm{trans}} = \delta m_{\tilde{q}}^{2\ \mathrm{trans}} = 0. \label{eq:SquarkSC}
\end{align}

\subsubsection{The Glunio Self-Energy}
The 4-spinor of the gluino comprises two Weyl spinors which describe very different particles.
\begin{align}
\tilde{g}^a = \begin{pmatrix}
\lambda^a \\
\overline{\chi}^a
\end{pmatrix}
\end{align}
The left-handed part $\lambda^a$ is associated with the superpartner of the gluon and therefore the "actual" gluino whereas the right-handed part $\overline{\chi}^a$ was introduced to assign a Dirac-mass to the gluino and is referred to as the octino.\\
From the Lagrangian \ref{eq:L_RSQCD} one can see that the couplings of the two particles are quite distinct. This is reflected by different field renormalization constants of the left- and right-handed part of the gluino.
\begin{figure}[!htbp]
\begin{center}
\begin{tikzpicture}[line width=1.5 pt, scale=1.3]
	\draw[fermionnoarrow](180:1.3)--(180:0.5);
	\draw[gluon](180:1.3)--(180:0.5);
	\draw[fermionnoarrow] (-0.5,0) arc (180:0:.5);
	\draw[fermionnoarrow] (0.5,0) arc (0:-180:.5);	
	\node at (0,0) {1L};
	\draw[fermionnoarrow](0:0.5)--(0:1.3);
	\draw[gluon](0:0.5)--(0:1.3);
	\node at (1.8,0) {=};
\begin{scope}[shift={(3.5,0)}]
	\draw[fermionnoarrow](180:1.3)--(180:0.5);
	\draw[gluon](180:1.3)--(180:0.5);
	\draw[fermionbar] (-0.5,0) arc (180:0:.5);
	\node at (0,0.8) {$q$};
	\draw[scalarbar] (0.5,0) arc (0:-180:.5);
	\node at (0,-0.8) {$\tilde{q}$};	
	\draw[fermionnoarrow](0:0.5)--(0:1.3);
	\draw[gluon](0:0.5)--(0:1.3);
	\node at (1.8,0) {+};
\end{scope}
\begin{scope}[shift={(7.0,0)}]
	\draw[gluon](180:1.3)--(180:0.5);
	\draw[fermionnoarrow](180:1.3)--(180:0.5);
	\draw[scalarnoarrow] (-0.5,0) arc (180:0:.5);
	\node at (0,0.8) {$\phi^0,\sigma^0$};
	\draw[fermionnoarrow] (0.5,0) arc (0:-180:.5);
	\draw[gluon] (0.5,0) arc (0:-180:.5);
	\node at (0,-0.8) {$\tilde{g}$};	
	\draw[gluon](0:0.5)--(0:1.3);
	\draw[fermionnoarrow](0:0.5)--(0:1.3);
	\node at (1.8,0) {+};
\end{scope}
\begin{scope}[shift={(10.5,0)}]
	\draw[gluon](180:1.3)--(180:0.5);
	\draw[fermionnoarrow](180:1.3)--(180:0.5);
	\draw[gluon] (-0.5,0) arc (180:0:.5);
	\node at (0,0.8) {$G$};
	\draw[fermionnoarrow] (0.5,0) arc (0:-180:.5);	
	\draw[gluon] (0.5,0) arc (0:-180:.5);
	\node at (0,-0.8) {$\tilde{g}$};	
	\draw[fermionnoarrow](0:0.5)--(0:1.3);
	\draw[gluon](0:0.5)--(0:1.3);
\end{scope}
\end{tikzpicture}
\caption{diagrammatic contributions to the self-energy of the squark at 1-loop level}
\end{center}
\end{figure}
As for the quarks the fermion (and momentum) flow is from the right to the left.
\begin{align}
i\Gamma^{1L}_{\tilde{g}^a \overline{\tilde{g}}^b} &= i \frac{g_s^2}{16\pi^2}\delta_{ab} \left[-4T(F)\left( (n_f-1) B_1(p^2,0,m_{\tilde{q}}^2) + B_1(p^2,m_t^2,m_{\tilde{q}}^2) \right)P_L \slashed{p} \right.\nonumber\\
& + C(A)\left( (B_0(p^2,m_{\tilde{g}}^2,m_{\phi^0}^2) - B_0(p^2,m_{\tilde{g}}^2,m_{\sigma^0}^2))m_{\tilde{g}} - (B_1(p^2,m_{\tilde{g}}^2,m_{\phi^0}^2) + B_1(p^2,m_{\tilde{g}}^2,m_{\sigma^0}^2))\slashed{p} \right) \nonumber\\
&\left.+ C(A)\left( (2 - 4 B_0(p^2,0,m_{\tilde{g}}^2)) m_{\tilde{g}} - (1-2(B_0(p^2,0,m_{\tilde{g}}^2) + B_1(p^2,0,m_{\tilde{g}}^2)))\slashed{p} \right)\right]
\end{align}
Where $n_f = 6$ is the number of quark flavors.\\
The counterterm Feynman rule reads\\
\begin{tikzpicture}[line width=1.0 pt, scale=0.8]
	\node at (-2.5,-0.1) {$i\Gamma^{1L,ct}_{\tilde{g}^a \overline{\tilde{g}}^b}\ \hat{=}\ a$};
\begin{scope}[shift={(0.1,0)}]	
	\draw[gluon] (-1.3,0) --(1.3,0);
	\draw[fermionnoarrow] (-0.3,0.3) -- (0.3,-0.3);
	\draw[fermionnoarrow] (-0.3,-0.3) -- (0.3,0.3);
	\draw[fermionnoarrow] (-1.3,0) --(1.3,0);
	\node at (1.6,0) {$b$};
	\node at (7.8,0) {$\hat{=}\ i\delta_{ab}\left[ (\delta Z_{\tilde{g}}^L P_L + \delta Z_{\tilde{g}}^R P_R)\slashed{p} - \left(\frac{\delta Z_{\tilde{g}}^L+\delta Z_{\tilde{g}}^R}{2} m_{\tilde{g}} +\delta m_{\tilde{g}}\right) \right]. $};	
\end{scope}
\end{tikzpicture}\\
The on-shell renormalization conditions for the fields are
\begin{align}
& \frac{\partial}{\partial (P_L\slashed{p})} \left[ \Re (\Gamma^{\mathrm{1L}}_{\tilde{g}^a \overline{\tilde{g}}^b}) + \Gamma^{\mathrm{1L,ct}}_{\tilde{g}^a \overline{\tilde{g}}^b}  \right]_{\slashed{p} = m_{\tilde{g}}} = 0
&& \frac{\partial}{\partial (P_R\slashed{p})} \left[ \Re (\Gamma^{\mathrm{1L}}_{\tilde{g}^a \overline{\tilde{g}}^b}) + \Gamma^{\mathrm{1L,ct}}_{\tilde{g}^a \overline{\tilde{g}}^b}  \right]_{\slashed{p} = m_{\tilde{g}}} = 0
\end{align}
where the derivative of $\Sigma = \Sigma^{VL}P_L\slashed{p} + \Sigma^{VR}P_R\slashed{p} + \Sigma^{SL}P_L + \Sigma^{SR}P_R$ with respect to $P_A\slashed{p}$ \\
($A = L,R$) is defined by
\begin{align}
\left.\frac{\partial}{\partial (P_A\slashed{p})} \Sigma\right|_{\slashed{p} = m} = \Sigma^{VA} + \frac{\partial}{\partial p^2} \left( m^2 \Sigma^{VL} + m^2 \Sigma^{VR} + m \Sigma^{SL} + m \Sigma^{SR}\right).
\end{align}
This leads to the following renormalization constants
\begin{align}
\delta Z_{\tilde{g}}^L &= \frac{g_s^2}{16\pi^2}\Re \left[ 4T(F)\left( (n_f-1) B_1(m_{\tilde{g}}^2,0,m_{\tilde{q}}^2) + B_1(m_{\tilde{g}}^2,m_t^2,m_{\tilde{q}}^2) \right) \right.\nonumber\\
&+C(A) (B_1(m_{\tilde{g}}^2,m_{\tilde{g}}^2,m_{\phi^0}^2) + B_1(m_{\tilde{g}}^2,m_{\tilde{g}}^2,m_{\sigma^0}^2))\nonumber\\
&+C(A)(1-2(B_0(m_{\tilde{g}}^2,0,m_{\tilde{g}}^2) + B_1(m_{\tilde{g}}^2,0,m_{\tilde{g}}^2)))\nonumber\\
&+ 4T(F) m_{\tilde{g}}^2 \frac{\partial}{\partial p^2} \left( ((n_f-1))  B_1(p^2,0,m_{\tilde{q}}^2) +  B_1(p^2,m_t^2,m_{\tilde{q}}^2) \right)\nonumber\\
&-2 C(A) m_{\tilde{g}} \frac{\partial}{\partial p^2} \left( B_0(p^2,m_{\tilde{g}}^2,m_{\phi^0}^2)-B_0(p^2,m_{\tilde{g}}^2,m_{\sigma^0}^2) - B_1(p^2,m_{\tilde{g}}^2,m_{\phi^0}^2) - B_1(p^2,m_{\tilde{g}}^2,m_{\sigma^0}^2) \right)\nonumber\\
&-4 C(A) m_{\tilde{g}}^2 \frac{\partial}{\partial p^2} \left.\left( -B_0(p^2,0,m_{\tilde{g}}^2) + B_0(p^2,0,m_{\tilde{g}}^2) \right)\right]_{p^2=m_{\tilde{g}}^2}
\end{align}
and 
\begin{align}
\delta Z_{\tilde{g}}^R &= \frac{g_s^2}{16\pi^2}\Re \left[C(A) (B_1(m_{\tilde{g}}^2,m_{\tilde{g}}^2,m_{\phi^0}^2) + B_1(m_{\tilde{g}}^2,m_{\tilde{g}}^2,m_{\sigma^0}^2)) \right.\nonumber\\
&+C(A)(1-2(B_0(m_{\tilde{g}}^2,0,m_{\tilde{g}}^2) + B_1(m_{\tilde{g}}^2,0,m_{\tilde{g}}^2)))\nonumber\\
&+ 4T(F) m_{\tilde{g}}^2 \frac{\partial}{\partial p^2} \left( ((n_f-1))  B_1(p^2,0,m_{\tilde{q}}^2) +  B_1(p^2,m_t^2,m_{\tilde{q}}^2) \right)\nonumber\\
&-2 C(A) m_{\tilde{g}} \frac{\partial}{\partial p^2} \left( B_0(p^2,m_{\tilde{g}}^2,m_{\phi^0}^2)-B_0(p^2,m_{\tilde{g}}^2,m_{\sigma^0}^2) - B_1(p^2,m_{\tilde{g}}^2,m_{\phi^0}^2) - B_1(p^2,m_{\tilde{g}}^2,m_{\sigma^0}^2) \right)\nonumber\\
&-4 C(A) m_{\tilde{g}}^2 \frac{\partial}{\partial p^2} \left.\left( -B_0(p^2,0,m_{\tilde{g}}^2) + B_0(p^2,0,m_{\tilde{g}}^2) \right)\right]_{p^2=m_{\tilde{g}}^2}.
\end{align}
As for the quark there are constant terms amid the Passarino-Veltman integrals. These arise only in DREG and not in DRED. The transition counterterms are
\begin{align}
\delta Z_{\tilde{g}}^{A\ \mathrm{trans}} = \delta Z_{\tilde{g}}^{A\ \mathrm{DREG}} - \delta Z_{\tilde{g}}^{A\ \mathrm{DRED}} = C(A) \frac{g_s^2}{16\pi^2}\label{eq:GluinoSC}
\end{align}
for $A = L,R$. The gluino mass counterterm is ascertained by the condition
\begin{align}
\left[ \Re (\Gamma^{\mathrm{1L}}_{\tilde{g}^a \overline{\tilde{g}}^b}) + \Gamma^{\mathrm{1L,ct}}_{\tilde{g}^a \overline{\tilde{g}}^b}  \right]_{\slashed{p} = m_{\tilde{g}}} = 0
\end{align}
which is equivalent to 
\begin{align}
\delta m_{\tilde{g}} = \Re\left( m_{\tilde{g}}\frac{\Sigma^{VL}+\Sigma^{VR}}{2} + \frac{\Sigma^{SL}+\Sigma^{SR}}{2} \right)
\end{align}
and yields
\begin{align}
\delta m_{\tilde{g}} &= \frac{g_s^2}{16\pi^2} m_{\tilde{g}}^2\ \Re \left[ -2T(F) \left( (n_f-1)B_1(m_{\tilde{g}}^2,0,m_{\tilde{q}}^2) + B_1(m_{\tilde{g}}^2,m_t^2,m_{\tilde{q}}^2) \right) \right.\nonumber\\
& + C(A) \left( B_0(m_{\tilde{g}}^2,m_{\tilde{g}}^2,m_{\phi^0}^2) - B_0(m_{\tilde{g}}^2,m_{\tilde{g}}^2,m_{\sigma^0}^2) - B_1(m_{\tilde{g}}^2,m_{\tilde{g}}^2,m_{\phi^0}^2) -
B_1(m_{\tilde{g}}^2,m_{\tilde{g}}^2,m_{\sigma^0}^2) \right)\nonumber\\
& + C(A) \left.\left( 1 - 2 B_0(m_{\tilde{g}}^2,0,m_{\tilde{g}}^2) + 2 B_1(m_{\tilde{g}}^2,0,m_{\tilde{g}}^2) \right)\right].
\end{align}
Again there is a transition counterterm 
\begin{align}
\delta m_{\tilde{g}}^{\mathrm{trans}} = \delta m_{\tilde{g}}^{\mathrm{DREG}} - \delta m_{\tilde{g}}^{\mathrm{DRED}} = C(A) \frac{g_s^2}{16\pi^2} m_{\tilde{g}}.
\end{align}


\subsection{Renormalization of the Gauge Coupling}
The gauge coupling $g_s$ is renormalized in the $\overline{\mathrm{MS}}$-scheme with the modification that additional logarithms are substracted. This is to decouple heavy particles from the running of $\alpha_s = \frac{g_s^2}{4\pi}$. This renormalization procedure allows to adopt the experimental values of $\alpha_s$ from the PDF's. The running due to effects of heavy particles is then encoded in the logarithms of $\delta g_s$.\\
Extracting $\delta g_s$ from the quark-quark-gluon vertex requires not only the computation of $i\Gamma^{1L}_{q_i\overline{q}_jG^a}$ but also the (re)evaluation  of the auxiliary field renormalization constants $\delta Z_q^{\mathrm{aux}}$ and $\delta Z_G^{\mathrm{aux}}$. These will not be the same as the on-shell field renormalization.

\subsubsection{The Quark Self-Energy Revisited}
The quark self-energy has two contribution which are shown in figure \ref{fig:QuarkSelfEnergy}. The first one corresponds to light particles the second one to heavy particles: $i\Gamma^{\mathrm{1L}}_{q_i\overline{q}_j} = i\Gamma^{\mathrm{1L,light}}_{q_i\overline{q}_j} + i\Gamma^{\mathrm{1L,heavy}}_{q_i\overline{q}_j}$. For light particles only the UV-divergent part is kept. For heavy particles also the $\mu$-dependent terms are kept:
\begin{align}
i\Gamma^{\mathrm{1L,light}}_{q_i\overline{q}_j}|_{\mathrm{UV-div}} &= i \frac{g_s^2}{16\pi^2}\delta_{ij} \slashed{p} C(F) \frac{1}{\epsilon_{\mathrm{UV}}}\\
i\Gamma^{\mathrm{1L,heavy}}_{q_i\overline{q}_j}|_{\mathrm{UV-div,\mu-dep}} &= i \frac{g_s^2}{16\pi^2}\delta_{ij} \slashed{p} C(F)\left( \frac{1}{\epsilon_{\mathrm{UV}}} -\ln\frac{m_{\tilde{q}}^2}{\mu^2} \right)
\end{align}
The renormalization constant for the evaluation of $\delta g_s$ is determined by the condition
\begin{align}
\frac{\partial}{\partial \slashed{p}} \left[  \Gamma^{\mathrm{1L,light}}_{q_i\overline{q}_j}|_{\mathrm{UV-div}} + \Gamma^{\mathrm{1L,heavy}}_{q_i\overline{q}_j}|_{\mathrm{UV-div,\mu-dep}} + \Gamma^{\mathrm{1L,ct}}_{q_i\overline{q}_j}  \right]_{p^2 = 0} = 0
\end{align}
and computes to
\begin{align}
\delta Z_q^{\mathrm{aux}} = -\frac{g_s^2}{16\pi^2} C(F)\left[\left( \frac{1}{\epsilon_{\mathrm{UV}}} \right) + \left( \frac{1}{\epsilon_{\mathrm{UV}}} - \ln\frac{m_{\tilde{q}}^2}{\mu^2} \right) \right]
\end{align}
The first curved bracket corresponds to contributions from light particles in the loop whereas the second curved bracket corresponds to contributions from heavy particles in the loop.

\subsubsection{The Gluon Self-Energy}
As for the quark self-energy there are again contributions to the self-energy originating from light and heavy particles. As for the quark only specific terms are kept in calculating the auxiliary renormalization constants.  
\begin{figure}[!htbp]
\begin{center}
\begin{tikzpicture}[line width=1.5 pt, scale=1.3]
	\draw[gluon](180:1.3)--(180:0.5);
	\draw[fermionnoarrow] (-0.5,0) arc (180:0:.5);
	\draw[fermionnoarrow] (0.5,0) arc (0:-180:.5);	
	\node at (0,0.2) {1L};
	\node at (0,-0.18) {light};
	\draw[gluon](0:0.5)--(0:1.3);
	\node at (1.8,0) {=};
\begin{scope}[shift={(3.5,0)}]
	\draw[gluon](-1.3,0)--(0,0);
	\draw[gluon](0,0)--(1.3,0);
	\draw[gluon] (0,0) arc (-90:270:.5);
	\node at (0,1.3) {$G$};
	\node at (1.8,0) {+};
\end{scope}
\begin{scope}[shift={(7.0,0)}]
	\draw[gluon](180:1.3)--(180:0.5);
	\draw[fermion] (-0.5,0) arc (180:0:.5);
	\node at (0,0.8) {light $q$};
	\draw[fermion] (0.5,0) arc (0:-180:.5);
	\node at (0,-0.8) {light $q$};	
	\draw[gluon](0:0.5)--(0:1.3);
\end{scope}
\begin{scope}[shift={(10.5,0)}]
	\draw[gluon](180:1.3)--(180:0.5);
	\draw[ghost] (-0.5,0) arc (180:0:.5);
	\node at (0,0.8) {$\eta$};
	\draw[ghost] (0.5,0) arc (0:-180:.5);
	\node at (0,-0.8) {$\eta$};	
	\draw[gluon](0:0.5)--(0:1.3);
\end{scope}
\begin{scope}[shift={(3.5,-2.5)}]
	\node at (-1.8,0) {+};
	\draw[gluon](180:1.3)--(180:0.5);
	\draw[gluon] (-0.5,0) arc (180:0:.5);
	\node at (0,0.8) {$G$};
	\draw[gluon] (0.5,0) arc (0:-180:.5);
	\node at (0,-0.8) {$G$};	
	\draw[gluon](0:0.5)--(0:1.3);
\end{scope}
\end{tikzpicture}
\caption{contribution to the self-energy of the gluon originating from light particles}\label{fig:lightGluonSelfEnergy}
\end{center}
\end{figure}
\begin{align}
i\Gamma^{\mathrm{1L,light}}_{G_\mu^a G_\nu^b}|_{\mathrm{UV-div}} &= i\frac{g_s^2}{16\pi^2 \epsilon_{\mathrm{UV}}}\delta_{ab} \left[ 0 - \frac{4 (n_{f}-1)}{3} T(F) (p^2g^{\mu\nu}-p^\mu p^\nu) + \frac{C(A)}{12}(p^2g^{\mu\nu} + 2 p^\mu p^\nu) \right.\nonumber\\
&+\left. \frac{C(A)}{12}(19 p^2g^{\mu\nu} - 22 p^\mu p^\nu) \right]\nonumber\\
&=i\frac{g_s^2}{16\pi^2 \epsilon_{\mathrm{UV}}}\delta_{ab} \left[ - \frac{4 (n_{f}-1)}{3} T(F) + \frac{5}{3} C(A) \right](p^2g^{\mu\nu}-p^\mu p^\nu)
\end{align}
\begin{figure}[!htbp]
\begin{center}
\begin{tikzpicture}[line width=1.5 pt, scale=1.3]
	\draw[gluon](180:1.3)--(180:0.5);
	\draw[fermionnoarrow] (-0.5,0) arc (180:0:.5);
	\draw[fermionnoarrow] (0.5,0) arc (0:-180:.5);	
	\node at (0,0.2) {1L};
	\node at (0,-0.18) {heavy};
	\draw[gluon](0:0.5)--(0:1.3);
	\node at (1.8,0) {=};
\begin{scope}[shift={(3.5,0)}]
	\draw[gluon](-1.3,0)--(0,0);
	\draw[gluon](0,0)--(1.3,0);
	\draw[scalar] (0,0) arc (-90:270:.5);
	\node at (0,1.3) {$\tilde{q}$};
	\node at (1.8,0) {+};
\end{scope}
\begin{scope}[shift={(7.0,0)}]
	\draw[gluon](-1.3,0)--(0,0);
	\draw[gluon](0,0)--(1.3,0);
	\draw[scalarnoarrow] (0,0) arc (-90:270:.5);
	\node at (0,1.3) {$\phi^0,\sigma^0$};
	\node at (1.8,0) {+};
\end{scope}
\begin{scope}[shift={(10.5,0)}]
	\draw[gluon](180:1.3)--(180:0.5);
	\draw[fermion] (-0.5,0) arc (180:0:.5);
	\node at (0,0.8) {heavy $q$};
	\draw[fermion] (0.5,0) arc (0:-180:.5);
	\node at (0,-0.8) {heavy $q$};	
	\draw[gluon](0:0.5)--(0:1.3);
\end{scope}
\begin{scope}[shift={(3.5,-2.5)}]
	\node at (-1.8,0) {+};
	\draw[gluon](180:1.3)--(180:0.5);
	\draw[gluon] (-0.5,0) arc (180:0:.5);
	\draw[fermionnoarrow] (-0.5,0) arc (180:0:.5);
	\node at (0,0.8) {$\tilde{g}$};
	\draw[gluon] (0.5,0) arc (0:-180:.5);
	\draw[fermionnoarrow] (0.5,0) arc (0:-180:.5);
	\node at (0,-0.8) {$\tilde{g}$};	
	\draw[gluon](0:0.5)--(0:1.3);
	\node at (1.8,0) {+};
\end{scope}
\begin{scope}[shift={(7,-2.5)}]
	\draw[gluon](180:1.3)--(180:0.5);
	\draw[scalar] (-0.5,0) arc (180:0:.5);
	\node at (0,0.8) {$\tilde{q}$};
	\draw[scalar] (0.5,0) arc (0:-180:.5);
	\node at (0,-0.8) {$\tilde{q}$};	
	\draw[gluon](0:0.5)--(0:1.3);
	\node at (1.8,0) {+};
\end{scope}
\begin{scope}[shift={(10.5,-2.5)}]
	\draw[gluon](180:1.3)--(180:0.5);
	\draw[scalarnoarrow] (-0.5,0) arc (180:0:.5);
	\node at (0,0.8) {$\phi^0,\sigma^0$};
	\draw[scalarnoarrow] (0.5,0) arc (0:-180:.5);
	\node at (0,-0.8) {$\phi^0,\sigma^0$};	
	\draw[gluon](0:0.5)--(0:1.3);
\end{scope}
\end{tikzpicture}
\caption{contribution to the self-energy of the gluon originating from heavy particles, in the last diagram either $\phi^0$ or $\sigma^0$ are running in the loop}
\end{center}
\end{figure}
The heavy particle contributions are given by\\
\begin{align}
i\Gamma^{\mathrm{1L,heavy}}_{G_\mu^a G_\nu^b}|_{\mathrm{UV-div, \mu-dep}} &= i\frac{g_s^2}{16\pi^2}\delta_{ab} \left[ - 4 T(F)n_f \left( \frac{1}{\epsilon_{\mathrm{UV}}} - \ln \frac{m_{\tilde{q}}^2}{\mu2} \right)m_{\tilde{q}}^2 g^{\mu\nu} - C(A) \left( \frac{1}{\epsilon_{\mathrm{UV}}} - \ln \frac{m_{\phi^0}^2}{\mu2} \right)m_{\phi^0}^2 g^{\mu\nu} \right.\nonumber\\
&- C(A) \left( \frac{1}{\epsilon_{\mathrm{UV}}} - \ln \frac{m_{\sigma^0}^2}{\mu2} \right)m_{\sigma^0}^2 g^{\mu\nu} - \frac{4}{3}T(F)\left( \frac{1}{\epsilon_{\mathrm{UV}}} - \ln \frac{m_t^2}{\mu2} \right)(p^2g^{\mu\nu}-p^\mu p^\nu)\nonumber\\
& - \frac{4}{3} C(A) \left( \frac{1}{\epsilon_{\mathrm{UV}}} - \ln \frac{m_{\tilde{g}}^2}{\mu2} \right)(p^2g^{\mu\nu}-p^\mu p^\nu)\nonumber\\
& -\frac{2}{3}T(F) n_f \left( \frac{1}{\epsilon_{\mathrm{UV}}} - \ln \frac{m_{\tilde{q}}^2}{\mu2} \right)(p^2g^{\mu\nu}-p^\mu p^\nu) + 4T(F)n_f \left( \frac{1}{\epsilon_{\mathrm{UV}}} - \ln \frac{m_{\tilde{q}}^2}{\mu2} \right)m_{\tilde{q}}^2 g^{\mu\nu} \nonumber\\
&-\frac{1}{6} C(A) \left( \frac{1}{\epsilon_{\mathrm{UV}}} - \ln \frac{m_{\phi^0}^2}{\mu2} \right)(p^2g^{\mu\nu}-p^\mu p^\nu) + C(A) \left( \frac{1}{\epsilon_{\mathrm{UV}}} - \ln \frac{m_{\phi^0}^2}{\mu2} \right)m_{\phi^0}^2 g^{\mu\nu} \nonumber\\
&\left.-\frac{1}{6} C(A) \left( \frac{1}{\epsilon_{\mathrm{UV}}} - \ln \frac{m_{\phi^0}^2}{\mu2} \right)(p^2g^{\mu\nu}-p^\mu p^\nu) + C(A) \left( \frac{1}{\epsilon_{\mathrm{UV}}} - \ln \frac{m_{\phi^0}^2}{\mu2} \right)m_{\phi^0}^2 g^{\mu\nu}\right]\nonumber\\
&= i\frac{g_s^2}{16\pi^2}\delta_{ab} \left[- \frac{4}{3}T(F)\left( \frac{1}{\epsilon_{\mathrm{UV}}} - \ln \frac{m_t^2}{\mu2} \right)(p^2g^{\mu\nu}-p^\mu p^\nu)\right.\nonumber\\
& - \frac{4}{3} C(A) \left( \frac{1}{\epsilon_{\mathrm{UV}}} - \ln \frac{m_{\tilde{g}}^2}{\mu2} \right)(p^2g^{\mu\nu}-p^\mu p^\nu)\nonumber\\
& -\frac{2}{3}T(F) n_f \left( \frac{1}{\epsilon_{\mathrm{UV}}} - \ln \frac{m_{\tilde{q}}^2}{\mu2} \right)(p^2g^{\mu\nu}-p^\mu p^\nu) \nonumber\\
&-\frac{1}{6} C(A) \left( \frac{1}{\epsilon_{\mathrm{UV}}} - \ln \frac{m_{\phi^0}^2}{\mu2} \right)(p^2g^{\mu\nu}-p^\mu p^\nu)  \nonumber\\
&\left.-\frac{1}{6} C(A) \left( \frac{1}{\epsilon_{\mathrm{UV}}} - \ln \frac{m_{\phi^0}^2}{\mu2} \right)(p^2g^{\mu\nu}-p^\mu p^\nu) \right]
\end{align}
The counterterm Feynman rule for the gluon propagator is\\
\\
\begin{tikzpicture}[line width=1.0 pt, scale=0.8]
	\node at (-2.5,-0.1) {$i\Gamma^{\mathrm{1L,ct}}_{G_\mu^a G_\nu^b}\ \hat{=}\ a,\mu$};
\begin{scope}[shift={(0.4,0)}]	
	\draw[gluon] (-1.3,0) --(1.3,0);
	\draw[fermionnoarrow] (-0.3,0.3) -- (0.3,-0.3);
	\draw[fermionnoarrow] (-0.3,-0.3) -- (0.3,0.3);
	\node at (1.8,0) {$b,\nu$};
	\node at (5.6,0) {$\hat{=}\ -i \delta Z_G\left(p^2 g^{\mu\nu} - p^\mu p^\nu \right)\delta_{ab}.$};	
\end{scope}
\end{tikzpicture}\\
The renormalization condition for $\delta Z^{\mathrm{aux}}_G$ reads
\begin{align}
 \Gamma^{\mathrm{1L,light}}_{G_\mu^a G_\nu^b}|_{\mathrm{UV-div}} + \Gamma^{\mathrm{1L,heavy}}_{G_\mu^a G_\nu^b}|_{\mathrm{UV-div,\mu-dep}} - \delta Z_G^{\mathrm{aux}}\left(p^2 g^{\mu\nu} - p^\mu p^\nu \right)\delta_{ab} = 0
\end{align}
and yields
\begin{align}
\delta Z^{aux}_G &= \frac{g_s^2}{16\pi^2} \left\{\left[-\frac{4}{3}T(F)(n_f-1) + \frac{5}{3} C(A) \right] \frac{1}{\epsilon_{\mathrm{UV}}} + \left[ - \frac{4}{3}T(F) \left( \frac{1}{\epsilon_{\mathrm{UV}}} - \ln \frac{m_t^2}{\mu^2} \right)  \right.\right.\nonumber\\
&- \frac{4}{3} C(A) \left( \frac{1}{\epsilon_{\mathrm{UV}}} - \ln \frac{m_{\tilde{g}}^2}{\mu^2} \right) -\frac{2}{3}T(F) n_f \left( \frac{1}{\epsilon_{\mathrm{UV}}} - \ln \frac{m_{\tilde{q}}^2}{\mu^2} \right)\nonumber\\
& - \frac{1}{6} C(A) \left( \frac{1}{\epsilon_{\mathrm{UV}}} - \ln \frac{m_{\phi^0}^2}{\mu^2} \right) - \frac{1}{6} C(A) \left( \frac{1}{\epsilon_{\mathrm{UV}}} - \ln \frac{m_{\sigma^0}^2}{\mu^2} \right).
\end{align}



\subsubsection{The $q\overline{q}G$ Vertex Correction}
\begin{figure}[!htbp]
\begin{center}
\begin{tikzpicture}[line width=1.5 pt, scale=1.3]
	\draw[gluon](180:1.3)--(180:0.5);
	\draw[fermionnoarrow] (-0.5,0) arc (180:0:.5);
	\draw[fermionnoarrow] (0.5,0) arc (0:-180:.5);	
	\node at (0,0.2) {1L};
	\node at (0,-0.18) {light};
	\draw[fermionbar](45:0.5)--(45:1.3);
	\draw[fermionbar](-45:0.5)--(-45:1.3);
	\node at (1.8,0) {=};
\begin{scope}[shift={(3.5,0)}]
	\draw[gluon](180:1.3)--(180:0.5);
	\draw[fermionbar](180:0.5)--(45:0.5);
	\node at (0,0.5) {$q$};
	\draw[fermionbar](180:0.5)--(-45:0.5);
	\node at (0,-0.5) {$q$};
	\draw[gluon](45:0.5)--(-45:0.5);
	\node at (0.7,0) {$G$};
	\draw[fermionbar](45:0.5)--(45:1.3);
	\draw[fermionbar](-45:0.5)--(-45:1.3);
	\node at (1.8,0) {+};
\end{scope}
\begin{scope}[shift={(7,0)}]
	\draw[gluon](180:1.3)--(180:0.5);
	\draw[gluon](180:0.5)--(45:0.5);
	\node at (0,0.5) {$G$};
	\draw[gluon](180:0.5)--(-45:0.5);
	\node at (0,-0.5) {$G$};
	\draw[fermion](45:0.5)--(-45:0.5);
	\node at (0.7,0) {$q$};
	\draw[fermionbar](45:0.5)--(45:1.3);
	\draw[fermionbar](-45:0.5)--(-45:1.3);
\end{scope}
\end{tikzpicture}
\caption{contribution from light particles to the $q\overline{q}G$ vertex correction}\label{fig:GaugeCouplingCorrection}
\end{center}
\end{figure}
\begin{align}
i\Gamma^{\mathrm{1L,light}}_{q_i \overline{q}_j G_\mu^a}|_{\mathrm{UV-div}} &= -i g_s T^a_{ij} \gamma^\mu \frac{g_s^2}{16\pi^2 \epsilon_{\mathrm{UV}}} \left[ \left(C(F) -\frac{C(A)}{2}\right) + \frac{3}{2} C(A) \right]
\end{align}
\begin{figure}[!htbp]
\begin{center}
\begin{tikzpicture}[line width=1.5 pt, scale=1.3]
	\draw[gluon](180:1.3)--(180:0.5);
	\draw[fermionnoarrow] (-0.5,0) arc (180:0:.5);
	\draw[fermionnoarrow] (0.5,0) arc (0:-180:.5);	
	\node at (0,0.2) {1L};
	\node at (0,-0.18) {heavy};
	\draw[fermionbar](45:0.5)--(45:1.3);
	\draw[fermionbar](-45:0.5)--(-45:1.3);
	\node at (1.8,0) {=};
\begin{scope}[shift={(3.5,0)}]
	\draw[gluon](180:1.3)--(180:0.5);
	\draw[scalarbar](180:0.5)--(45:0.5);
	\node at (0,0.5) {$\tilde{q}$};
	\draw[scalarbar](180:0.5)--(-45:0.5);
	\node at (0,-0.5) {$\tilde{q}$};
	\draw[gluon](45:0.5)--(-45:0.5);
	\draw[fermionnoarrow](45:0.5)--(-45:0.5);
	\node at (0.7,0) {$\tilde{g}$};
	\draw[fermionbar](45:0.5)--(45:1.3);
	\draw[fermionbar](-45:0.5)--(-45:1.3);
	\node at (1.8,0) {+};
\end{scope}
\begin{scope}[shift={(7,0)}]
	\draw[gluon](180:1.3)--(180:0.5);
	\draw[gluon](180:0.5)--(45:0.5);
	\draw[fermionnoarrow](180:0.5)--(45:0.5);
	\node at (0,0.5) {$\tilde{g}$};
	\draw[gluon](180:0.5)--(-45:0.5);
	\draw[fermionnoarrow](180:0.5)--(-45:0.5);
	\node at (0,-0.5) {$\tilde{g}$};
	\draw[scalar](45:0.5)--(-45:0.5);
	\node at (0.7,0) {$\tilde{q}$};
	\draw[fermionbar](45:0.5)--(45:1.3);
	\draw[fermionbar](-45:0.5)--(-45:1.3);
\end{scope}
\end{tikzpicture}
\caption{contribution from heavy particles to the $q\overline{q}G$ vertex correction}
\end{center}
\end{figure}
\begin{align}
i\Gamma^{\mathrm{1L,heavy}}_{q_i \overline{q}_j G_\mu^a}|_{\mathrm{UV-div,\mu-dep}} = -i g_s T^a_{ij} \gamma^\mu \frac{g_s^2}{16\pi^2} &\left[ \left(C(F) -\frac{C(A)}{2}\right) \left( \frac{1}{\epsilon_{\mathrm{UV}}} - \ln \frac{m_{\tilde{g}}^2}{\mu^2} \right) \right.\nonumber\\
&+ \left.\frac{1}{2} C(A)\left( \frac{1}{\epsilon_{\mathrm{UV}}} - \ln \frac{m_{\tilde{g}}^2}{\mu^2} \right) \right]
\end{align}
The sum of the 1-loop corrections and the counterterm should be set to zero:
\begin{align}
i\Gamma^{\mathrm{1L,light}}_{q_i \overline{q}_j G_\mu^a}|_{\mathrm{UV-div}} + i\Gamma^{\mathrm{1L,heavy}}_{q_i \overline{q}_j G_\mu^a}|_{\mathrm{UV-div,\mu-dep}} + \left[ -ig_s T^a_{ij}\gamma^\mu\left( \frac{\delta g_s}{g_s} + \delta Z_q^{\mathrm{aux}} + \frac{\delta Z_G^{\mathrm{aux}}}{2} \right)\right] = 0.
\end{align}
Finally one can read off the $\frac{\delta g_s}{g_s}$
\begin{align}
\frac{\delta g_s}{g_s} &= \frac{g_s^2}{16\pi^2} \left[ \left( \frac{2}{3}T(F)(n_f-1) - \frac{11}{6}C(A) \right) \frac{1}{\epsilon_{\mathrm{UV}}} + \left( \frac{5}{6}C(A) + \frac{2}{3}T(F) + \frac{1}{3}T(F)n_f \right)\frac{1}{\epsilon_{\mathrm{UV}}}  \right.\nonumber\\
&- \frac{2}{3} C(A) \ln \frac{m_{\tilde{g}}^2}{\mu^2} - \frac{1}{3}T(F)n_f \ln \frac{m_{\tilde{q}}^2}{\mu^2} - \frac{2}{3}T(F) \ln \frac{m_t^2}{\mu^2}-\frac{1}{12} C(A) \left( \ln \frac{m_{\phi^0}^2}{\mu^2} + \ln \frac{m_{\sigma^0}^2}{\mu^2} \right)\label{eq:deltaGs}
\end{align}

\subsubsection{The Beta Function}
The beta function describes the dependence of the gauge coupling $g_s$ upon the energy scale $\mu$.\\
Writing down the action of a theory in $D$ dimensions one needs to introduce an energy scale $\mu$ in order to keep the action dimensionless. But $\mu$ is no physical parameter and can be absorped into the fields and parameters. To this end one defines
\begin{align}
g_{sB} = \mu^\epsilon g_s \left( 1 + \frac{\delta g_s}{g_s} \right)
\end{align}
which must not depend upon the unphysical scale $\mu$, ergo
\begin{align}
0 = \frac{\mathrm{d}g_{sB}}{\mathrm{d}\ln\mu} = \frac{\partial g_{sB}}{\partial \ln\mu} + \beta \frac{\partial g_{sB}}{\partial g_s}\label{eq:beta_func}
\end{align}
where the definition of the beta function $\frac{\partial g_s}{\partial\ln\mu}$ has been inserted. Equation \ref{eq:beta_func} serves to calculate $\beta(g_s,\epsilon)$.  By equating coefficients and using the shortcuts
\begin{align*}
\frac{\beta_0^L}{2} &= \frac{2}{3}T(F)(n_f-1) - \frac{11}{6}C(A)\\
\frac{\beta_0^H}{2} &= \frac{5}{6}C(A) + \frac{2}{3}T(F) + \frac{1}{3}T(F)n_f\\
L &= - \frac{2}{3} C(A) \ln \frac{m_{\tilde{g}}^2}{\mu^2} - \frac{1}{3}T(F)n_f \ln \frac{m_{\tilde{q}}^2}{\mu^2} - \frac{2}{3}T(F) \ln \frac{m_t^2}{\mu^2}-\frac{1}{12} C(A) \left( \ln \frac{m_{\phi^0}^2}{\mu^2} + \ln \frac{m_{\sigma^0}^2}{\mu^2} \right)
\end{align*}
so that 
\begin{align}
\frac{\delta g_s}{g_s} = \frac{g_s^2}{16\pi^2}\left( \frac{\beta^L_0}{2\epsilon_{\mathrm{UV}}} + \frac{\beta^H_0}{2\epsilon_{\mathrm{UV}}} + L \right)
\end{align}
one finds
\begin{align}
\beta(g_s,\epsilon) &= -\epsilon g_s \left( 1 + \frac{g_s^2}{16\pi^2} L \right) + \beta(g_s) + \mathcal{O}(\mathrm{2-loop})\\
\beta(g_s) &= \frac{g_s^3}{16\pi^2} \beta_0^L.
\end{align}
This is exactely the beta function from QCD first found by [Gross, Politzer, Wil]


\subsection{Supersymmetry Restoring Counterterm}
As already discussed in section \ref{sec:RegSchemeDep} care is required in terms of supersymmetry restoring when renormalizing  the gauge coupling $g_s$ and the Yukawa coupling $\hat{g}_s$. In doing so one needs the already calculated supersymmetry restoring counterterms of the quark, squark and gluino from \ref{eq:QuarkSC}, \ref{eq:SquarkSC} and \ref{eq:GluinoSC}
as well as the supersymmetry restoring counterterm of the gluon.
\subsubsection*{The Gluon Self-Energy Revisited}
The only regularization dependence of the gluon self-energy arises from the gluon loop, i.e. the last diagram in figure \ref{fig:lightGluonSelfEnergy}. With the definition of $\Gamma^{\mathrm{(1),ct,restore}}_{\mathrm{DREG}}$ in \ref{eq:GammaCtRestore} one obtains
\begin{align}
i\Gamma^{\mathrm{(1),ct,restore}}_{\mathrm{DREG},G_\mu^a G_\nu^b} = -i\frac{1}{3} C(A) \frac{g_s^2}{16\pi^2}(p^2 g^{\mu\nu}-p^\mu p^\nu)\delta_{ab}
\end{align}
which translates to the transition counterterm
\begin{align}
\delta Z^{\mathrm{trans}}_G = \frac{C(A)}{3}\frac{g_s^2}{16\pi^2}.
\end{align}

\subsubsection*{The $q\overline{q}G$ Vertex Correction Revisited}
The supersymmetry restoring contributions to the gauge coupling correction are shown in figure \ref{fig:GaugeCouplingCorrection} and evaluate to
\begin{align}
i\Gamma^{\mathrm{(1),ct,restore}}_{\mathrm{DREG}, q_i\overline{q}_jG_\mu^a} &= -ig_s T^a_{ij} \gamma^\mu \frac{g_s^2}{16\pi^2}\left[ \left( C(F) - \frac{C(A)}{2} \right) + \frac{C(A)}{2} \right]\\
&= -ig_s T^a_{ij} \gamma^\mu \left[ \frac{\delta g_s^{\mathrm{trans}}}{g_s} + \delta Z^{\mathrm{trans}}_q + \frac{\delta Z^{\mathrm{trans}}_G}{2} \right]
\end{align}
where in the second line the equation with the supersymmetry restoring counterterms has been performed. This yields
\begin{align}
\frac{\delta g_s^{\mathrm{trans}}}{g_s} = -\frac{C(A)}{6} \frac{g_s^2}{16\pi^2}.
\end{align}

\subsubsection*{The $q\tilde{q}^\dagger\tilde{g}$ Vertex Correction}
The supersymmetry restoring corrections to the Yukawa coupling origin from the below diagram 
\begin{figure}[!htbp]
\begin{center}
\begin{tikzpicture}[line width=1.5 pt, scale=1.3]
	\draw[gluon](180:1.3)--(180:0.5);
	\draw[fermionnoarrow](180:1.3)--(180:0.5);
	\draw[gluon](180:0.5)--(45:0.5);
	\node at (0,0.5) {$G$};
	\draw[fermionnoarrow](180:0.5)--(-45:0.5);
	\draw[gluon](180:0.5)--(-45:0.5);
	\node at (0,-0.5) {$\tilde{g}$};
	\draw[scalar](45:0.5)--(-45:0.5);
	\node at (0.7,0) {$q$};
	\draw[fermionbar](45:0.5)--(45:1.3);
	\draw[scalarbar](-45:0.5)--(-45:1.3);
\end{tikzpicture}
\caption{diagram of the supersymmetry restoring correction of  the $q\tilde{q}\tilde{g}$ vertex}
\end{center}
\end{figure}
The supersymmetry restoring part is
\begin{align}
i\Gamma^{\mathrm{(1),ct,restore}}_{\mathrm{DREG}, q_i\tilde{q}_j\tilde{g}^a} &= -ig_s \sqrt{2} P_L T^a_{ij}  \frac{g_s^2}{16\pi^2}C(A)\\
&= -ig_s \sqrt{2} P_L T^a_{ij}  \left[ \frac{\delta \hat{g}_s^{\mathrm{trans}}}{g_s} + \frac{\delta Z^{\mathrm{trans}}_q + \delta Z^{\mathrm{trans}}_{\tilde{q}} + \delta Z^{\mathrm{trans}}_{\tilde{g}}}{2} \right].
\end{align}
The supersymmetry restoring part of the Yukawa renormalization constants is therefore
\begin{align}
\frac{\delta \hat{g}_s^{\mathrm{trans}}}{g_s} = -\frac{C(F)-C(A)}{2} \frac{g_s^2}{16\pi^2}.
\end{align}
As a consequence of the two different supersymmetry restoring parts of the coupling renormalization constants an additional renormalization constant $\delta g_s^{\mathrm{restore}}$ needs to be introduced. As described in section \label{sec:RegSchemeDep} it is given by
\begin{align}
\frac{\delta g_s^{\mathrm{restore}}}{g_s} = \frac{\delta \hat{g}_s^{\mathrm{trans}}}{g_s} -\frac{\delta g_s^{\mathrm{trans}}}{g_s} = \frac{g_s^2}{16\pi^2}\left( \frac{2C(A)}{3} - \frac{C(F)}{2} \right).
\end{align}
In short this means that the gauge coupling $g_s$ is renormalized with $\delta g_s$ given in \ref{eq:deltaGs} and the Yukawa coupling $\hat{g}_s$ is renormilzed with $\delta \hat{g}_s = \delta g_s + \delta g_s^{\mathrm{restore}}$.\\
The finite correction $g_s^{\mathrm{restore}}$ is the same as in supersymmetric QCD which should not surprise too much as all its contributions origin from loops with gluons. So there are no new contributions in RSQCD with respect to SQCD. 


\subsection{$\overline{\mathrm{MS}}$ - Renormalization}
To check for UV-finiteness it might prove useful to summarize the UV-divergent part of all renormalization constants. In order to obtain these the Passarino-Veltman integrals need to be substituted by their $\frac{1}{\epsilon}$ coefficient. These  had been taken from \cite{Denner} and checked with \texttt{FeynArts} and \texttt{FormCalc} \cite{Hahn:2000}, \cite{Nejad2013}, \cite{Hahn:2000}.