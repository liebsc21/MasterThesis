% Differential quotients
\newcommand{\tdfrac} [2]{\ensuremath{\frac{\mathrm{d}   #1}{\mathrm{d} {#2}}}}%
\newcommand{\tddfrac}[2]{\ensuremath{\frac{\mathrm{d}^2 #1}{\mathrm{d} {#2}^2}}}%
\newcommand{\pdfrac} [2]{\ensuremath{\frac{\partial     #1}{\partial   {#2}}}}%
\newcommand{\pddfrac}[2]{\ensuremath{\frac{\partial^2   #1}{\partial   {#2}^2}}}%
\newcommand{\varfrac} [2]{\ensuremath{\frac{\delta     #1}{\delta   {#2}}}}%

% Merke-Umgebung
\usepackage{shadethm}
\newshadetheorem{remark_}{Merke}
\newenvironment{remark}[1][]{%
  \definecolor{shadethmcolor}{rgb}{.9,.9,.9}       % Farbe Hintergrund
  \definecolor{shaderulecolor}{rgb}{1.0,0.0,0.0}   % Farbe Rahmen
  \setlength{\shadeboxrule}{1pt}%
  \begin{remark_}[#1]%
}{\end{remark_}}

% Einschub-Umgebung
\newshadetheorem{insertion_}{Einschub}
\newenvironment{insertion}{
  \definecolor{shadethmcolor}{rgb}{0.9,0.8,.9}
  \definecolor{shaderulecolor}{rgb}{0.0,0.0,1}
  \setlength{\shadeboxrule}{1pt}
  \begin{insertion_}%\hspace*{1mm}
}{\end{insertion_}}

% Nebenrechnung-Umgebung
\newshadetheorem{auxcal_}{Nebenrechnung}
\newenvironment{auxcal}{
  \definecolor{shadethmcolor}{rgb}{0.7,0.9,0.9}
  \definecolor{shaderulecolor}{rgb}{0.0,0.1,0.1}
  \setlength{\shadeboxrule}{1pt}
  \begin{auxcal_}%\hspace*{1mm}
}{\end{auxcal_}}

% tikz-Packete
\usepackage{tikz}
\usetikzlibrary{arrows}
\usetikzlibrary{shapes,arrows}
\usetikzlibrary{positioning}
\usetikzlibrary{fit}
\usetikzlibrary{shadows}

% eigene Befehle für tikz
\tikzstyle{decision} = [diamond, draw, fill=blue!20, 
text width=4.5em, text badly centered, node distance=3cm, inner sep=0pt]
\tikzstyle{block1} = [rectangle, draw, fill=blue!20, 
text width=10em, text centered, rounded corners, minimum height=4em]
\tikzstyle{block2} = [rectangle, draw, fill=red!20, 
text width=10em, text centered, rounded corners, minimum height=4em]
\tikzstyle{line} = [draw, -latex']
\tikzstyle{cloud} = [draw, ellipse,fill=red!20, node distance=3cm, minimum height=2em]


% These are examples of Feynman Diagrams using TikZ
% Author: Flip Tanedo, 2010. pt267@cornell.edu
%\documentclass[12pt]{article}	% Why do I use such a boring document clas?
% THE USUAL PACKAGES
\usepackage{amsmath,amssymb,amsfonts}	% DON'T use cite, screws with bibtex + hyperref
\usepackage{color}
\usepackage{amsmath}
\usepackage{amsfonts}
\usepackage{amssymb}
\usepackage{graphicx}
\usepackage{slashed}            % for slashed characters in math mode
\usepackage{bbm}                % for \mathbbm{1} (unit matrix)
\usepackage{xspace}				% For spacing after commands
% TIKZ - for drawing Feynman diagrams
% ... use with pdflatex
\usepackage{tikz}
\usetikzlibrary{arrows,shapes}
\usetikzlibrary{trees}
\usetikzlibrary{matrix,arrows} 				% For commutative diagram
											% http://www.felixl.de/commu.pdf
\usetikzlibrary{positioning}				% For "above of=" commands
\usetikzlibrary{calc,through}				% For coordinates
\usetikzlibrary{decorations.pathreplacing}  % For curly braces
% http://www.math.ucla.edu/~getreuer/tikz.html
\usepackage{pgffor}							% For repeating patterns
\usetikzlibrary{decorations.pathmorphing}	% For Feynman Diagrams
\usetikzlibrary{decorations.markings}
\tikzset{
	% >=stealth', %%  Uncomment for more conventional arrows
    vector/.style={decorate, decoration={snake}, draw},
	provector/.style={decorate, decoration={snake,amplitude=2.5pt}, draw},
	antivector/.style={decorate, decoration={snake,amplitude=-2.5pt}, draw},
    fermion/.style={draw=black, postaction={decorate},
        decoration={markings,mark=at position .55 with {\arrow[draw=black]{>}}}},   
    fermionbar/.style={draw=black, postaction={decorate},
        decoration={markings,mark=at position .55 with {\arrow[draw=black]{<}}}},
    fermionnoarrow/.style={draw=black},
    gluon/.style={decorate, draw=black,
        decoration={coil,amplitude=4pt, segment length=5pt}},
    scalar/.style={dashed,draw=black, postaction={decorate},
        decoration={markings,mark=at position .55 with {\arrow[draw=black]{>}}}},
    scalarbar/.style={dashed,draw=black, postaction={decorate},
        decoration={markings,mark=at position .55 with {\arrow[draw=black]{<}}}},
    scalarnoarrow/.style={dashed,draw=black},
    electron/.style={draw=black, postaction={decorate},
        decoration={markings,mark=at position .55 with {\arrow[draw=black]{>}}}},
	bigvector/.style={decorate, decoration={snake,amplitude=4pt}, draw},
	ghost/.style={dotted,draw=black, postaction={decorate},
        decoration={markings,mark=at position .55 with {\arrow[draw=black]{>}}}},
	ghostbar/.style={dotted,draw=black, postaction={decorate},
        decoration={markings,mark=at position .55 with {\arrow[draw=black]{<}}}},
}
% TIKZ - for block diagrams, 
% from http://www.texample.net/tikz/examples/control-system-principles/
% \usetikzlibrary{shapes,arrows}
\tikzstyle{block} = [draw, rectangle, 
    minimum height=3em, minimum width=6em]
% circled numbers
\newcommand*\circled[1]{\tikz[baseline=(char.base)]{
            \node[shape=circle,draw,inner sep=2pt] (char) {#1};}}
%% selbst definierte Befehle für tikz
\tikzstyle{block1a} = [rectangle, draw, fill=blue!20, 
text width=10em, text centered, rounded corners, minimum height=1cm]
\tikzstyle{block2a} = [rectangle, draw, fill=red!20, 
text width=10em, text centered, rounded corners, minimum height=1cm]
\tikzstyle{block1b} = [rectangle, draw, fill=blue!40, 
text width=10em, text centered, rounded corners, minimum height=1cm]
\tikzstyle{block2b} = [rectangle, draw, fill=red!40, 
text width=10em, text centered, rounded corners, minimum height=1cm]
\tikzstyle{block3} = [rectangle, draw, fill=green!40, 
text width=18em, text centered, rounded corners, minimum height=1cm]
\tikzstyle{block4} = [rectangle, draw, fill=blue!40, 
text width=10.2em, text centered, rounded corners, minimum height=1cm]
\tikzstyle{block5a} = [rectangle, draw, fill=red!40, 
text width=28.5em, text centered, rounded corners, minimum height=1cm]
\tikzstyle{block5b} = [rectangle, draw, fill=red!20, 
text width=8em, text centered, rounded corners, minimum height=1cm]
\tikzstyle{block5c} = [rectangle, draw, fill=red!20, 
text width=9em, text centered, rounded corners, minimum height=1cm]
\tikzstyle{block6a} = [rectangle, draw, fill=green!40, 
text width=15em, text centered, rounded corners, minimum height=1cm]
\tikzstyle{line} = [draw, -latex']       
\usepackage{hyperref}			% Has to be at the end.
