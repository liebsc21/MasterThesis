\section{Introduction}
The Standard Model of particle physics is a quantum field theory which relies on the principle of gauge invariance. It describes the world's fundamental particles and the interactions among them. It predicts the most accurately measured observables in physics and is therefore considered to be a very successful theory.\\
Besides these triumphs, it is known to be incomplete for it does, for example, not include dark matter. Because of these and other flaws, one studies extensions of the Standard Model. One mathematically very appealing extension is the concept of supersymmetry. However, besides the provision of a dark matter candidate and elegant solutions of, e.g. the hierarchy problem, supersymmetry does not come without problems. First of all, supersymmetric particles have not been discovered so far. Secondly, there are flaws like the ``SUSY flavor problem''. A non-trivial extension of supersymmetry - $R$-symmetry - may explain or solve these issues. Imposing supersymmetry and $R$-symmetry on the Standard Model, gives - according to the Haag-Lopuszanski-Sohnius theorem - the maximal symmetry group a reasonable quantum field theory can have.\\
In this thesis a minimal extension of the Standard Model including supersymmetry and $R$-symmetry - the MRSSM - will be considered. More specifically, the production cross section of strongly interacting supersymmetric particles is studied. It will turn out that there are significant changes in the production of squarks and gluinos when turning from the Minimal Supersymmetric Standard Model (MSSM) to the Minimal $R$-symmetric Supersymmetric Standard Model (MRSSM).\\
This thesis starts with summarizing the main aspects of the Standard Model before introducing the concept of supersymmetry and $R$-symmetry. Hereby, the relevant models including the minimal extension of the respective symmetry, the MSSM and the MRSSM are introduced. Section \ref{sec:SquarkGluinoTree} compares the production cross section of squark and gluinos in both models at tree-level. It will turn out that the most striking difference manifests in the channel of squark production. Therefore, the rest of the thesis focuses on the next-to-leading order calculation of this process: Section \ref{sec:VirtRealCorr} discusses the golden thread of the calculation of a next-to-leading order cross section including massless external particles before section \ref{sec:renMRSSM} gives detailed information about the renormalization of the MRSSM. This includes the explicit form of the required renormalization constants, reaching form the on-shell scheme over a mixture of $\overline{\mathrm{MS}}$- and zero-momentum subtraction scheme to supersymmetry restoring counterterms. This section also discusses what the most copious contributions to the renormalized matrix element are. The thesis closes with the representation of the next-to-leading order cross section for squark production in the form of $K$-factors.

%motivation: aesthetic: Coleman-Mandula --> Haag-Lopuszanski-Sohnius-Theorem\\
%plots for exclusion of squarks in specific SUSY scenarios (from Michael) --> R-Symmetry could be possible explanation for that because:\\
%MSSM-Lagrangian --> trafo rules for superfields under R-symm --> forbidden terms in MRSSM (write down Lagrangian for R-symmetric SUSYQCD)\\
%suppression of squark production in MRSSM by less diagrams ($m_{gluino}^{-4}$ suppression at low energies in MRSSM and only $m_{gluino}^{-2}$ suppression in MSSM)\\
%R-charges of all fields (show in diagram!) --> only if R-charges of final / initial particles are zero, a diagram is allowed in R-symm. model\\
%references to build in
%\begin{itemize}
%\item "Matching Squark Pair Production at NLO with Parton Showers" from Gavin, Hangst, Krämer, Mühlleitner,.. for complete treatment of NLO calculation\\
%\item "dIRAC gAUGINOS IN susy - sUPPRESSED jETS + <met sIGNALS: a sNOWMASS wHITEPAPER" FROM kRIBS, mARTIN for Squark production at LO, allude to same result\\
%\item "Dirac Gaugino Masses and supersoft SUSY breaking" from Fox, Weiner, Nelson for the introduction of the MRSSM
%\end{itemize}