\section{Introduction}
The Standard Model of particle physics is a quantum field theory which relies on the principle of gauge invariance. It describes the world's fundamental particles and the interactions among them and predicts the most accurately measured observables in physics.\\
Besides these triumphs, it is known to be incomplete for example for not including dark matter or a description of gravity. Because of these and other flaws, one studies extensions of the Standard Model. One mathematically very appealing extension is the concept of supersymmetry. Within this framework one often studies the Minimal Supersymmetric Standard Model (MSSM). This theory provides a candidate for dark matter and elegant solutions of e.g. the hierarchy problem. However the MSSM does not come without problems. One of these is that supersymmetric particles have not been discovered so far. This is  actually reasoned by introducing supersymmetry breaking. However, in order to still solve the hierarchy problem, supersymmetry breaking should not be too large. According to searches for supersymmetry at the Large Hadron Collider, supersymmetric particles are at the moment on the verge on being too heavy for the MSSM to solve the hierarchy problem without loosing its attractiveness.\\
A non-trivial extension of supersymmetry - $R$-symmetry - may explain this and other issues. $R$-symmetry is, according to the Haag-Lopuszanski-Sohnius theorem, the only viable extension of the supersymmetry group of the MSSM. This is, the combination of Poincaré symmetry, supersymmetry and $R$-symmetry is the maximal symmetry a reasonable quantum field theory can have.\\
\ignore{Imposing supersymmetry and $R$-symmetry on the Standard Model, gives - according to the Haag-Lopuszanski-Sohnius theorem - the maximal symmetry group a reasonable quantum field theory can have.\\}
Motivated be these facts, the Minimal $R$-Symmetric Supersymmetric Standard Model (MRSSM) is the subject of investigations within this thesis. More specifically, the production cross section of strongly interacting supersymmetric particles in the MRSSM is studied. It will turn out that there are significant changes in the production of squarks and gluinos when turning from the MSSM to the MRSSM.\\
This thesis starts with summarizing the main aspects of the Standard Model before the concept of supersymmetry and $R$-symmetry is introduced. This includes the introduction of the models which incorporate the minimal form of the symmetry in question. Section \ref{sec:SquarkGluinoTree} compares the production cross section of squark and gluinos in both models at \mbox{tree-level}. After identifying an attractive region of parameter space in the MRSSM, it will turn out that the most interesting difference manifests in the channel of squark production. As there are no programs available with whom next-to-leading order cross sections within the MRSSM can be calculated, the rest of the thesis focuses on the next-to-leading order calculation of the squark production cross section.
: Section \ref{sec:VirtRealCorr} discusses the golden thread of the calculation of a next-to-leading order cross section including massless external particles before section \ref{sec:renMRSSM} gives detailed information about the renormalization of the MRSSM. This includes the explicit form of required renormalization constants, reaching form the on-shell scheme over a mixture of $\overline{\mathrm{MS}}$- and zero-momentum subtraction scheme to supersymmetry restoring counterterms. This section also discusses how renormalized one-loop matrix elements are generated as an input for a program which calculates the squark production cross section in the MRSSM.
The thesis closes with an explanation on how this very program is constructed and the representation of the next-to-leading order cross section for squark production in the form of $K$-factors.

%motivation: aesthetic: Coleman-Mandula --> Haag-Lopuszanski-Sohnius-Theorem\\
%plots for exclusion of squarks in specific SUSY scenarios (from Michael) --> R-Symmetry could be possible explanation for that because:\\
%MSSM-Lagrangian --> trafo rules for superfields under R-symm --> forbidden terms in MRSSM (write down Lagrangian for R-symmetric SUSYQCD)\\
%suppression of squark production in MRSSM by less diagrams ($m_{gluino}^{-4}$ suppression at low energies in MRSSM and only $m_{gluino}^{-2}$ suppression in MSSM)\\
%R-charges of all fields (show in diagram!) --> only if R-charges of final / initial particles are zero, a diagram is allowed in R-symm. model\\
%references to build in
%\begin{itemize}
%\item "Matching Squark Pair Production at NLO with Parton Showers" from Gavin, Hangst, Krämer, Mühlleitner,.. for complete treatment of NLO calculation\\
%\item "dIRAC gAUGINOS IN susy - sUPPRESSED jETS + <met sIGNALS: a sNOWMASS wHITEPAPER" FROM kRIBS, mARTIN for Squark production at LO, allude to same result\\
%\item "Dirac Gaugino Masses and supersoft SUSY breaking" from Fox, Weiner, Nelson for the introduction of the MRSSM
%\end{itemize}