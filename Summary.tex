\section{Summary}

The Standard Model of particle physics describes the world's fundamental particles and the interactions among. It belongs to the best tested theory within physics but is on the other hand known to be incomplete.\\
Supersymmetry provides an attractive extension of the Standard Model. Within this thesis, a specific supersymmetric extension of the Standard Model - the Minimal $R$-Symmetric Supersymmetric Standard Model (MRSSM) - is studied. This is particularly appealing from the point of view of symmetries because it comprises the maximal possible symmetry group, a reasonable quantum field theory can have. The theorem stating this, is the Haag-Laopuszanski-Sohnius theorem. In contrast to the common notion, supersymmetry alone does not extent the Poincaré symmetry to the largest possible symmetry group. There is a further non-trivial extension by a global continuous symmetry, referred to as $R$-symmetry.\\
This thesis introduces an $R$-symmetric model in its minimal form. $R$-symmetry forbids terms which are allowed in the Minimal Supersymmetric Standard Model (MSSM). This includes the $A$-terms and the $\mu$-term which are responsible for squark mixing and flavor changing processes. But $R$-symmetry also prohibits Majorana mass terms for the gauginos. To be phenomenological viable, the MRSSM needs to accomodate Dirac gauginos. This requires the gauge sector of this model to be a $N=2$ supersymmetric theory which in turn implies further degrees of freedom.\\
Focusing on the strongly coupling sector, there is, apart from the ``usual'' gluino, another fermionic degree of freedom which is referred to the octino. The superpartner of this right handed part of the strongly coupling gaugino is a spin-0 particle: the scalar gluon (sgluon). As a matter of fact, there are two distinct real sgluons with distinct masses.\\ 
Within the framework of this thesis, the production cross section of squarks and gluinos in the MRSSM has been calculated at tree-level. To this end, Feynman rules for the strongly coupling sector of the MRSSM have been derived. The cross sections and intermediate results are given explicitly for each process. In comparison with the corresponding processes in the MSSM, it has been found that gluino production is enhanced by about a factor of two for gluino and antigluino are distinguishable particles, as opposed to in the MSSM. In order to meet the absence of signals alluding to supersymmetry at the LHC, this suggests that gluinos might be heavier than squarks if they should be realized in that way in nature. Focusing on this regime of parameter space, it has been found that the production of squarks and gluinos in the MRSSM is significantly lower, than in the MSSM. The most striking difference has been found in the production channel of squarks. Motivated by this fact, the next-to-leading order cross section for squark production has been calculated.\\
As this had not been done so far, it was not possible to use a tool like \texttt{MadGraph} or \texttt{Prospino} to perform the calculation. Instead, an own program needed to be devised to calculate the next-to-leading order cross section of squark production in the MRSSM in an automated way.\\
To this end a model file, generated by \texttt{Sarah}, which serves only for leading order processes has been adjusted manually to incorporate counter term Feynman rules. At this point, Philip Dießner did a major contribution when the model file was changed. After this, the model file has been provided with appropriate renormalization constants which are explicitly given in the thesis. This includes also a supersymmetry restoring counterterm. By means of this model file, renormalized one-loop corrections to the matrix element of squark production has been calculated in \texttt{Mathematica}, using the packages \texttt{FeynArts} and \texttt{FormCalc}. After this, the virtual contributions, which have been checked for ultraviolet-finiteness have been passed to a \texttt{C++} code and have been evaluated using the \texttt{Looptools} package. Finally, the phase space integration has been performed numerically using the \texttt{CUBA} library for parallelized numerical integration. However, the virtual corrections to the cross section are not infrared finite. To render the next-to-leading order cross section $\sigma^{\mathrm{NLO}}$ finite, real corrections need to be included. This has been done by Wojciech Kotlarski. After including them, $\sigma^{\mathrm{NLO}}$ has been found to contain no further divergences.
\begin{table}[H]
\begin{center}
\begin{tabular}{c|c?c|c|c?c|c|c}
\multicolumn{2}{c?}{} & \multicolumn{3}{c?}{MRSSM: $p + p \to \tilde{u}_L + \tilde{u}_R$} & \multicolumn{3}{c}{MSSM: $p + p \to \tilde{u} + \tilde{u}$}\\
\hlinewd{2pt}
$m_{\tilde{q}}$ in GeV & $m_{\tilde{g}}$ in GeV & $\sigma^{\mathrm{LO}}$ in fb & $\sigma^{\mathrm{NLO}}$ in fb & $K$ & $\sigma^{\mathrm{LO}}$ in fb & $\sigma^{\mathrm{NLO}}$ in fb & $K$\\
\hlinewd{2pt}
$500$ & $500$ & $966.4$ & $1441$ & $1.49$ & $2333$ & $3102$ & $1.33$\\
$500$ & $1000$ & $303.4$ & $383$ & $1.26$ & $1271$ & $1393$ & $1.10$\\
$1000$ & $1000$ & $42.63$ & $61.8$ & $1.45$ & $122.2$ & $161.3$ & $1.32$\\
$1000$ & $2000$ & $11.56$ & $14.3$ & $1.24$ & $6.36$ & $72.1$ & $1.09$\\
$1500$ & $3000$ & $0.9166$ & $1.13$ & $1.23$ & $7.045$ & $7.57$ & $1.07$
\end{tabular}
\caption{Total hadronic cross section for the production of up-squarks in the MRSSM and the MSSM at leading order and next-to-leading order. Also given are the $K$-factors. As a consequence of $R$-charge conservation, the only allowed channel in the MRSSM is $\tilde{u}_L + \tilde{u}_R$ production, whereas in the MSSM also $\tilde{u}_L + \tilde{u}_L$ and $\tilde{u}_R + \tilde{u}_R$ production is allowed.\newline 
The results are shown for a selected set of masses. The center-of-mass energy is $\sqrt{S} = \unit[13]{TeV}$ and within the MRSSM the pseudoscalar mass is fixed to $m_{\mathrm{\sigma}} = \unit[5000]{GeV}$.}\label{tab:MRSSMKfactors}
\end{center}
\end{table}
Equipped with this program, the $K$-factors for up-squark production have been calculated as a function of the squark mass and a function of the gluino mass. As a result, the comparison of $K$-factors for squark production in the MRSSM and the MSSM showed that the very same are slightly larger in the MRSSM. But since the tree-level cross section of squark production in the MRSSM are, depending on the choice  of squark and gluino mass, about an order of magnitude smaller than in the MSSM, this renders the MRSSM to a perfectly viable extension of the Standard Model.\\
Among other advantages to the MSSM, it explains a suppressed cross section for the production of supersymmetric particles at the LHC.

