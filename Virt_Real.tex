\section{Virtual and Real Corrections}
This section describes the necessary steps in the calculation of the squark production cross section at next-to-leading-order. The $\mathcal{O}(\alpha_s)$ correction to the tree level process includes the computation of one-loop diagrams such as shown in fig. ???. To yield $\mathcal{O}(\alpha_s)$ corrections in the cross section the interference term between the one-loop amplitude $\mathcal{M}^{\mathrm{1L}}$ and the Born amplitude $\mathcal{M}^{\mathrm{B}}$ needs to be considered. However the virtual amplitude 
\begin{align}
\mathcal{M}^{\mathrm{V}} = \mathcal{M}^{\mathrm{B}} \mathcal{M}^{\mathrm{1L}\ast} + \mathcal{M}^{\mathrm{1L}} \mathcal{M}^{\mathrm{B}\ast} = 2 \Re \left( \mathcal{M}^{\mathrm{B}} \mathcal{M}^{\mathrm{1L}\ast} \right)
\end{align}
contains divergences. These come from integrals over undetermined momentum and energy of particles running in loops. Due to their origin these divergences are referred to as ultraviolet and infrared divergences.
Ultraviolet divergences occur when the loop momentum tends to infinity which corresponds to arbitrary short distance interactions. However, these divergences are cured by first regularizing them and then introducing appropriate counterterms in the Lagrangian to cancel the extracted singularities. The second step is called renormalization and can be understood as a redefinition or rescaling of parameters and fields in the Lagrangian in the first place. This procedure will be discussed in detail in section \label{sec:renMRSSM}.\\
Having removed the ultraviolet divergences one performs the 2-body phase space integration to arrive at the cross section. However, this is not finite as it comprises infrared divergences. These split into soft and collinear (or mass) singularities\footnote{The names differ in the literature. Often infrared divergences are used as a synonym for soft divergences.} which cannot be removed by means of renormalization.


Naunberg 


\subsection{Virtual Correction}