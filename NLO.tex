\section{Squark Production at One-Loop}


\subsection{The LSZ-Theorem}


\subsection{The Cross Section in the Limit of High Sgluon Masses}
The cross section for squark production does not exist in the limit of an infinitely large sgluon mass, instead it was found that it diverges logarithmically.\\
\begin{align}
\lim_{m_{\sigma^0}\to\infty} \sigma(qq \to \tilde{q}\tilde{q}) \sim \ln \frac{m_{\sigma^0}^2}{\mu^2}
\end{align}
This is actually expected as an effective field theory of the MRSSM where the sgluon is integrated out is no longer supersymmetric. This is because the sgluon is together with the octino part of a supermultiplet. Integrating out only the sgluon means that the octino misses its superpartner in the effective field theory. In this case the decoupling theorem \cite{Appelquist:1974tg} does no longer hold.\\
Refer to super oblique correction and quantify difference of $g$ and $\hat{g}$ from eq 4 in \cite{Cheng:1997sq}\\
append plot